\documentclass[10pt]{scrartcl}

    % included at the top of the generated TeX file

\usepackage{tgcursor}

\usepackage[utf8]{inputenc}

\KOMAoptions{parskip=half*}
\KOMAoptions{paper=a4,twoside=false}
\KOMAoptions{numbers=noendperiod}

\addtokomafont{disposition}{\rmfamily}
\setcounter{tocdepth}{\subsectiontocdepth}


\usepackage{etoolbox}
\makeatletter
\providecommand{\institute}[1]{% add institute to \maketitle
  \apptocmd{\@author}{\end{tabular}
    \par
    \begin{tabular}[t]{c}
    \textit{#1}}{}{}
}
\makeatother

    \usepackage[breakable]{tcolorbox}
    \usepackage{parskip} % Stop auto-indenting (to mimic markdown behaviour)
    
    \usepackage{iftex}
    \ifPDFTeX
    	\usepackage[T1]{fontenc}
    	\usepackage{mathpazo}
    \else
    	\usepackage{fontspec}
    \fi

    % Basic figure setup, for now with no caption control since it's done
    % automatically by Pandoc (which extracts ![](path) syntax from Markdown).
    \usepackage{graphicx}
    % Maintain compatibility with old templates. Remove in nbconvert 6.0
    \let\Oldincludegraphics\includegraphics
    % Ensure that by default, figures have no caption (until we provide a
    % proper Figure object with a Caption API and a way to capture that
    % in the conversion process - todo).
    \usepackage{caption}
    \DeclareCaptionFormat{nocaption}{}
    \captionsetup{format=nocaption,aboveskip=0pt,belowskip=0pt}

    \usepackage{float}
    \floatplacement{figure}{H} % forces figures to be placed at the correct location
    \usepackage{xcolor} % Allow colors to be defined
    \usepackage{enumerate} % Needed for markdown enumerations to work
    \usepackage{geometry} % Used to adjust the document margins
    \usepackage{amsmath} % Equations
    \usepackage{amssymb} % Equations
    \usepackage{textcomp} % defines textquotesingle
    % Hack from http://tex.stackexchange.com/a/47451/13684:
    \AtBeginDocument{%
        \def\PYZsq{\textquotesingle}% Upright quotes in Pygmentized code
    }
    \usepackage{upquote} % Upright quotes for verbatim code
    \usepackage{eurosym} % defines \euro
    \usepackage[mathletters]{ucs} % Extended unicode (utf-8) support
    \usepackage{fancyvrb} % verbatim replacement that allows latex
    \usepackage{grffile} % extends the file name processing of package graphics 
                         % to support a larger range
    \makeatletter % fix for old versions of grffile with XeLaTeX
    \@ifpackagelater{grffile}{2019/11/01}
    {
      % Do nothing on new versions
    }
    {
      \def\Gread@@xetex#1{%
        \IfFileExists{"\Gin@base".bb}%
        {\Gread@eps{\Gin@base.bb}}%
        {\Gread@@xetex@aux#1}%
      }
    }
    \makeatother
    \usepackage[Export]{adjustbox} % Used to constrain images to a maximum size
    \adjustboxset{max size={0.9\linewidth}{0.9\paperheight}}

    % The hyperref package gives us a pdf with properly built
    % internal navigation ('pdf bookmarks' for the table of contents,
    % internal cross-reference links, web links for URLs, etc.)
    \usepackage{hyperref}
    % The default LaTeX title has an obnoxious amount of whitespace. By default,
    % titling removes some of it. It also provides customization options.
    \usepackage{titling}
    \usepackage{longtable} % longtable support required by pandoc >1.10
    \usepackage{booktabs}  % table support for pandoc > 1.12.2
    \usepackage[inline]{enumitem} % IRkernel/repr support (it uses the enumerate* environment)
    \usepackage[normalem]{ulem} % ulem is needed to support strikethroughs (\sout)
                                % normalem makes italics be italics, not underlines
    \usepackage{mathrsfs}
    

    
    % Colors for the hyperref package
    \definecolor{urlcolor}{rgb}{0,.145,.698}
    \definecolor{linkcolor}{rgb}{.71,0.21,0.01}
    \definecolor{citecolor}{rgb}{.12,.54,.11}

    % ANSI colors
    \definecolor{ansi-black}{HTML}{3E424D}
    \definecolor{ansi-black-intense}{HTML}{282C36}
    \definecolor{ansi-red}{HTML}{E75C58}
    \definecolor{ansi-red-intense}{HTML}{B22B31}
    \definecolor{ansi-green}{HTML}{00A250}
    \definecolor{ansi-green-intense}{HTML}{007427}
    \definecolor{ansi-yellow}{HTML}{DDB62B}
    \definecolor{ansi-yellow-intense}{HTML}{B27D12}
    \definecolor{ansi-blue}{HTML}{208FFB}
    \definecolor{ansi-blue-intense}{HTML}{0065CA}
    \definecolor{ansi-magenta}{HTML}{D160C4}
    \definecolor{ansi-magenta-intense}{HTML}{A03196}
    \definecolor{ansi-cyan}{HTML}{60C6C8}
    \definecolor{ansi-cyan-intense}{HTML}{258F8F}
    \definecolor{ansi-white}{HTML}{C5C1B4}
    \definecolor{ansi-white-intense}{HTML}{A1A6B2}
    \definecolor{ansi-default-inverse-fg}{HTML}{FFFFFF}
    \definecolor{ansi-default-inverse-bg}{HTML}{000000}

    % common color for the border for error outputs.
    \definecolor{outerrorbackground}{HTML}{FFDFDF}

    % commands and environments needed by pandoc snippets
    % extracted from the output of `pandoc -s`
    \providecommand{\tightlist}{%
      \setlength{\itemsep}{0pt}\setlength{\parskip}{0pt}}
    \DefineVerbatimEnvironment{Highlighting}{Verbatim}{commandchars=\\\{\}}
    % Add ',fontsize=\small' for more characters per line
    \newenvironment{Shaded}{}{}
    \newcommand{\KeywordTok}[1]{\textcolor[rgb]{0.00,0.44,0.13}{\textbf{{#1}}}}
    \newcommand{\DataTypeTok}[1]{\textcolor[rgb]{0.56,0.13,0.00}{{#1}}}
    \newcommand{\DecValTok}[1]{\textcolor[rgb]{0.25,0.63,0.44}{{#1}}}
    \newcommand{\BaseNTok}[1]{\textcolor[rgb]{0.25,0.63,0.44}{{#1}}}
    \newcommand{\FloatTok}[1]{\textcolor[rgb]{0.25,0.63,0.44}{{#1}}}
    \newcommand{\CharTok}[1]{\textcolor[rgb]{0.25,0.44,0.63}{{#1}}}
    \newcommand{\StringTok}[1]{\textcolor[rgb]{0.25,0.44,0.63}{{#1}}}
    \newcommand{\CommentTok}[1]{\textcolor[rgb]{0.38,0.63,0.69}{\textit{{#1}}}}
    \newcommand{\OtherTok}[1]{\textcolor[rgb]{0.00,0.44,0.13}{{#1}}}
    \newcommand{\AlertTok}[1]{\textcolor[rgb]{1.00,0.00,0.00}{\textbf{{#1}}}}
    \newcommand{\FunctionTok}[1]{\textcolor[rgb]{0.02,0.16,0.49}{{#1}}}
    \newcommand{\RegionMarkerTok}[1]{{#1}}
    \newcommand{\ErrorTok}[1]{\textcolor[rgb]{1.00,0.00,0.00}{\textbf{{#1}}}}
    \newcommand{\NormalTok}[1]{{#1}}
    
    % Additional commands for more recent versions of Pandoc
    \newcommand{\ConstantTok}[1]{\textcolor[rgb]{0.53,0.00,0.00}{{#1}}}
    \newcommand{\SpecialCharTok}[1]{\textcolor[rgb]{0.25,0.44,0.63}{{#1}}}
    \newcommand{\VerbatimStringTok}[1]{\textcolor[rgb]{0.25,0.44,0.63}{{#1}}}
    \newcommand{\SpecialStringTok}[1]{\textcolor[rgb]{0.73,0.40,0.53}{{#1}}}
    \newcommand{\ImportTok}[1]{{#1}}
    \newcommand{\DocumentationTok}[1]{\textcolor[rgb]{0.73,0.13,0.13}{\textit{{#1}}}}
    \newcommand{\AnnotationTok}[1]{\textcolor[rgb]{0.38,0.63,0.69}{\textbf{\textit{{#1}}}}}
    \newcommand{\CommentVarTok}[1]{\textcolor[rgb]{0.38,0.63,0.69}{\textbf{\textit{{#1}}}}}
    \newcommand{\VariableTok}[1]{\textcolor[rgb]{0.10,0.09,0.49}{{#1}}}
    \newcommand{\ControlFlowTok}[1]{\textcolor[rgb]{0.00,0.44,0.13}{\textbf{{#1}}}}
    \newcommand{\OperatorTok}[1]{\textcolor[rgb]{0.40,0.40,0.40}{{#1}}}
    \newcommand{\BuiltInTok}[1]{{#1}}
    \newcommand{\ExtensionTok}[1]{{#1}}
    \newcommand{\PreprocessorTok}[1]{\textcolor[rgb]{0.74,0.48,0.00}{{#1}}}
    \newcommand{\AttributeTok}[1]{\textcolor[rgb]{0.49,0.56,0.16}{{#1}}}
    \newcommand{\InformationTok}[1]{\textcolor[rgb]{0.38,0.63,0.69}{\textbf{\textit{{#1}}}}}
    \newcommand{\WarningTok}[1]{\textcolor[rgb]{0.38,0.63,0.69}{\textbf{\textit{{#1}}}}}
    
    
    % Define a nice break command that doesn't care if a line doesn't already
    % exist.
    \def\br{\hspace*{\fill} \\* }
    % Math Jax compatibility definitions
    \def\gt{>}
    \def\lt{<}
    \let\Oldtex\TeX
    \let\Oldlatex\LaTeX
    \renewcommand{\TeX}{\textrm{\Oldtex}}
    \renewcommand{\LaTeX}{\textrm{\Oldlatex}}
    % Document parameters
    % Document title
    \newcommand*{\mytitle}{Unit 6: Advanced NumPy}

    % Included at the bottom of the preamble

\usepackage{microtype}


\title{\mytitle}
\author{Richard Foltyn}
\hypersetup{pdfauthor={Richard Foltyn}, pdftitle={\mytitle}}

% needed to access graphs in lectures/solutions folder
\graphicspath{{../lectures/}}

% Remove horizontal rules in a very hackish way
\renewcommand{\rule}[2]{}

\RequirePackage{xspace}


\newcommand*{\eg}{e.g.\@\xspace}
\newcommand*{\Eg}{E.g.\@\xspace}
\newcommand*{\etc}{etc.\@\xspace}
\newcommand*{\ie}{i.e.\@\xspace}
\newcommand*{\vs}{vs.\@\xspace}
\newcommand*{\viz}{viz.\@\xspace}
\newcommand*{\US}{U.S.\@\xspace}


    
    
    
    
    
% Pygments definitions
\makeatletter
\def\PY@reset{\let\PY@it=\relax \let\PY@bf=\relax%
    \let\PY@ul=\relax \let\PY@tc=\relax%
    \let\PY@bc=\relax \let\PY@ff=\relax}
\def\PY@tok#1{\csname PY@tok@#1\endcsname}
\def\PY@toks#1+{\ifx\relax#1\empty\else%
    \PY@tok{#1}\expandafter\PY@toks\fi}
\def\PY@do#1{\PY@bc{\PY@tc{\PY@ul{%
    \PY@it{\PY@bf{\PY@ff{#1}}}}}}}
\def\PY#1#2{\PY@reset\PY@toks#1+\relax+\PY@do{#2}}

\expandafter\def\csname PY@tok@w\endcsname{\def\PY@tc##1{\textcolor[rgb]{0.73,0.73,0.73}{##1}}}
\expandafter\def\csname PY@tok@c\endcsname{\let\PY@it=\textit\def\PY@tc##1{\textcolor[rgb]{0.25,0.50,0.50}{##1}}}
\expandafter\def\csname PY@tok@cp\endcsname{\def\PY@tc##1{\textcolor[rgb]{0.74,0.48,0.00}{##1}}}
\expandafter\def\csname PY@tok@k\endcsname{\let\PY@bf=\textbf\def\PY@tc##1{\textcolor[rgb]{0.00,0.50,0.00}{##1}}}
\expandafter\def\csname PY@tok@kp\endcsname{\def\PY@tc##1{\textcolor[rgb]{0.00,0.50,0.00}{##1}}}
\expandafter\def\csname PY@tok@kt\endcsname{\def\PY@tc##1{\textcolor[rgb]{0.69,0.00,0.25}{##1}}}
\expandafter\def\csname PY@tok@o\endcsname{\def\PY@tc##1{\textcolor[rgb]{0.40,0.40,0.40}{##1}}}
\expandafter\def\csname PY@tok@ow\endcsname{\let\PY@bf=\textbf\def\PY@tc##1{\textcolor[rgb]{0.67,0.13,1.00}{##1}}}
\expandafter\def\csname PY@tok@nb\endcsname{\def\PY@tc##1{\textcolor[rgb]{0.00,0.50,0.00}{##1}}}
\expandafter\def\csname PY@tok@nf\endcsname{\def\PY@tc##1{\textcolor[rgb]{0.00,0.00,1.00}{##1}}}
\expandafter\def\csname PY@tok@nc\endcsname{\let\PY@bf=\textbf\def\PY@tc##1{\textcolor[rgb]{0.00,0.00,1.00}{##1}}}
\expandafter\def\csname PY@tok@nn\endcsname{\let\PY@bf=\textbf\def\PY@tc##1{\textcolor[rgb]{0.00,0.00,1.00}{##1}}}
\expandafter\def\csname PY@tok@ne\endcsname{\let\PY@bf=\textbf\def\PY@tc##1{\textcolor[rgb]{0.82,0.25,0.23}{##1}}}
\expandafter\def\csname PY@tok@nv\endcsname{\def\PY@tc##1{\textcolor[rgb]{0.10,0.09,0.49}{##1}}}
\expandafter\def\csname PY@tok@no\endcsname{\def\PY@tc##1{\textcolor[rgb]{0.53,0.00,0.00}{##1}}}
\expandafter\def\csname PY@tok@nl\endcsname{\def\PY@tc##1{\textcolor[rgb]{0.63,0.63,0.00}{##1}}}
\expandafter\def\csname PY@tok@ni\endcsname{\let\PY@bf=\textbf\def\PY@tc##1{\textcolor[rgb]{0.60,0.60,0.60}{##1}}}
\expandafter\def\csname PY@tok@na\endcsname{\def\PY@tc##1{\textcolor[rgb]{0.49,0.56,0.16}{##1}}}
\expandafter\def\csname PY@tok@nt\endcsname{\let\PY@bf=\textbf\def\PY@tc##1{\textcolor[rgb]{0.00,0.50,0.00}{##1}}}
\expandafter\def\csname PY@tok@nd\endcsname{\def\PY@tc##1{\textcolor[rgb]{0.67,0.13,1.00}{##1}}}
\expandafter\def\csname PY@tok@s\endcsname{\def\PY@tc##1{\textcolor[rgb]{0.73,0.13,0.13}{##1}}}
\expandafter\def\csname PY@tok@sd\endcsname{\let\PY@it=\textit\def\PY@tc##1{\textcolor[rgb]{0.73,0.13,0.13}{##1}}}
\expandafter\def\csname PY@tok@si\endcsname{\let\PY@bf=\textbf\def\PY@tc##1{\textcolor[rgb]{0.73,0.40,0.53}{##1}}}
\expandafter\def\csname PY@tok@se\endcsname{\let\PY@bf=\textbf\def\PY@tc##1{\textcolor[rgb]{0.73,0.40,0.13}{##1}}}
\expandafter\def\csname PY@tok@sr\endcsname{\def\PY@tc##1{\textcolor[rgb]{0.73,0.40,0.53}{##1}}}
\expandafter\def\csname PY@tok@ss\endcsname{\def\PY@tc##1{\textcolor[rgb]{0.10,0.09,0.49}{##1}}}
\expandafter\def\csname PY@tok@sx\endcsname{\def\PY@tc##1{\textcolor[rgb]{0.00,0.50,0.00}{##1}}}
\expandafter\def\csname PY@tok@m\endcsname{\def\PY@tc##1{\textcolor[rgb]{0.40,0.40,0.40}{##1}}}
\expandafter\def\csname PY@tok@gh\endcsname{\let\PY@bf=\textbf\def\PY@tc##1{\textcolor[rgb]{0.00,0.00,0.50}{##1}}}
\expandafter\def\csname PY@tok@gu\endcsname{\let\PY@bf=\textbf\def\PY@tc##1{\textcolor[rgb]{0.50,0.00,0.50}{##1}}}
\expandafter\def\csname PY@tok@gd\endcsname{\def\PY@tc##1{\textcolor[rgb]{0.63,0.00,0.00}{##1}}}
\expandafter\def\csname PY@tok@gi\endcsname{\def\PY@tc##1{\textcolor[rgb]{0.00,0.63,0.00}{##1}}}
\expandafter\def\csname PY@tok@gr\endcsname{\def\PY@tc##1{\textcolor[rgb]{1.00,0.00,0.00}{##1}}}
\expandafter\def\csname PY@tok@ge\endcsname{\let\PY@it=\textit}
\expandafter\def\csname PY@tok@gs\endcsname{\let\PY@bf=\textbf}
\expandafter\def\csname PY@tok@gp\endcsname{\let\PY@bf=\textbf\def\PY@tc##1{\textcolor[rgb]{0.00,0.00,0.50}{##1}}}
\expandafter\def\csname PY@tok@go\endcsname{\def\PY@tc##1{\textcolor[rgb]{0.53,0.53,0.53}{##1}}}
\expandafter\def\csname PY@tok@gt\endcsname{\def\PY@tc##1{\textcolor[rgb]{0.00,0.27,0.87}{##1}}}
\expandafter\def\csname PY@tok@err\endcsname{\def\PY@bc##1{\setlength{\fboxsep}{0pt}\fcolorbox[rgb]{1.00,0.00,0.00}{1,1,1}{\strut ##1}}}
\expandafter\def\csname PY@tok@kc\endcsname{\let\PY@bf=\textbf\def\PY@tc##1{\textcolor[rgb]{0.00,0.50,0.00}{##1}}}
\expandafter\def\csname PY@tok@kd\endcsname{\let\PY@bf=\textbf\def\PY@tc##1{\textcolor[rgb]{0.00,0.50,0.00}{##1}}}
\expandafter\def\csname PY@tok@kn\endcsname{\let\PY@bf=\textbf\def\PY@tc##1{\textcolor[rgb]{0.00,0.50,0.00}{##1}}}
\expandafter\def\csname PY@tok@kr\endcsname{\let\PY@bf=\textbf\def\PY@tc##1{\textcolor[rgb]{0.00,0.50,0.00}{##1}}}
\expandafter\def\csname PY@tok@bp\endcsname{\def\PY@tc##1{\textcolor[rgb]{0.00,0.50,0.00}{##1}}}
\expandafter\def\csname PY@tok@fm\endcsname{\def\PY@tc##1{\textcolor[rgb]{0.00,0.00,1.00}{##1}}}
\expandafter\def\csname PY@tok@vc\endcsname{\def\PY@tc##1{\textcolor[rgb]{0.10,0.09,0.49}{##1}}}
\expandafter\def\csname PY@tok@vg\endcsname{\def\PY@tc##1{\textcolor[rgb]{0.10,0.09,0.49}{##1}}}
\expandafter\def\csname PY@tok@vi\endcsname{\def\PY@tc##1{\textcolor[rgb]{0.10,0.09,0.49}{##1}}}
\expandafter\def\csname PY@tok@vm\endcsname{\def\PY@tc##1{\textcolor[rgb]{0.10,0.09,0.49}{##1}}}
\expandafter\def\csname PY@tok@sa\endcsname{\def\PY@tc##1{\textcolor[rgb]{0.73,0.13,0.13}{##1}}}
\expandafter\def\csname PY@tok@sb\endcsname{\def\PY@tc##1{\textcolor[rgb]{0.73,0.13,0.13}{##1}}}
\expandafter\def\csname PY@tok@sc\endcsname{\def\PY@tc##1{\textcolor[rgb]{0.73,0.13,0.13}{##1}}}
\expandafter\def\csname PY@tok@dl\endcsname{\def\PY@tc##1{\textcolor[rgb]{0.73,0.13,0.13}{##1}}}
\expandafter\def\csname PY@tok@s2\endcsname{\def\PY@tc##1{\textcolor[rgb]{0.73,0.13,0.13}{##1}}}
\expandafter\def\csname PY@tok@sh\endcsname{\def\PY@tc##1{\textcolor[rgb]{0.73,0.13,0.13}{##1}}}
\expandafter\def\csname PY@tok@s1\endcsname{\def\PY@tc##1{\textcolor[rgb]{0.73,0.13,0.13}{##1}}}
\expandafter\def\csname PY@tok@mb\endcsname{\def\PY@tc##1{\textcolor[rgb]{0.40,0.40,0.40}{##1}}}
\expandafter\def\csname PY@tok@mf\endcsname{\def\PY@tc##1{\textcolor[rgb]{0.40,0.40,0.40}{##1}}}
\expandafter\def\csname PY@tok@mh\endcsname{\def\PY@tc##1{\textcolor[rgb]{0.40,0.40,0.40}{##1}}}
\expandafter\def\csname PY@tok@mi\endcsname{\def\PY@tc##1{\textcolor[rgb]{0.40,0.40,0.40}{##1}}}
\expandafter\def\csname PY@tok@il\endcsname{\def\PY@tc##1{\textcolor[rgb]{0.40,0.40,0.40}{##1}}}
\expandafter\def\csname PY@tok@mo\endcsname{\def\PY@tc##1{\textcolor[rgb]{0.40,0.40,0.40}{##1}}}
\expandafter\def\csname PY@tok@ch\endcsname{\let\PY@it=\textit\def\PY@tc##1{\textcolor[rgb]{0.25,0.50,0.50}{##1}}}
\expandafter\def\csname PY@tok@cm\endcsname{\let\PY@it=\textit\def\PY@tc##1{\textcolor[rgb]{0.25,0.50,0.50}{##1}}}
\expandafter\def\csname PY@tok@cpf\endcsname{\let\PY@it=\textit\def\PY@tc##1{\textcolor[rgb]{0.25,0.50,0.50}{##1}}}
\expandafter\def\csname PY@tok@c1\endcsname{\let\PY@it=\textit\def\PY@tc##1{\textcolor[rgb]{0.25,0.50,0.50}{##1}}}
\expandafter\def\csname PY@tok@cs\endcsname{\let\PY@it=\textit\def\PY@tc##1{\textcolor[rgb]{0.25,0.50,0.50}{##1}}}

\def\PYZbs{\char`\\}
\def\PYZus{\char`\_}
\def\PYZob{\char`\{}
\def\PYZcb{\char`\}}
\def\PYZca{\char`\^}
\def\PYZam{\char`\&}
\def\PYZlt{\char`\<}
\def\PYZgt{\char`\>}
\def\PYZsh{\char`\#}
\def\PYZpc{\char`\%}
\def\PYZdl{\char`\$}
\def\PYZhy{\char`\-}
\def\PYZsq{\char`\'}
\def\PYZdq{\char`\"}
\def\PYZti{\char`\~}
% for compatibility with earlier versions
\def\PYZat{@}
\def\PYZlb{[}
\def\PYZrb{]}
\makeatother


    % For linebreaks inside Verbatim environment from package fancyvrb. 
    \makeatletter
        \newbox\Wrappedcontinuationbox 
        \newbox\Wrappedvisiblespacebox 
        \newcommand*\Wrappedvisiblespace {\textcolor{red}{\textvisiblespace}} 
        \newcommand*\Wrappedcontinuationsymbol {\textcolor{red}{\llap{\tiny$\m@th\hookrightarrow$}}} 
        \newcommand*\Wrappedcontinuationindent {3ex } 
        \newcommand*\Wrappedafterbreak {\kern\Wrappedcontinuationindent\copy\Wrappedcontinuationbox} 
        % Take advantage of the already applied Pygments mark-up to insert 
        % potential linebreaks for TeX processing. 
        %        {, <, #, %, $, ' and ": go to next line. 
        %        _, }, ^, &, >, - and ~: stay at end of broken line. 
        % Use of \textquotesingle for straight quote. 
        \newcommand*\Wrappedbreaksatspecials {% 
            \def\PYGZus{\discretionary{\char`\_}{\Wrappedafterbreak}{\char`\_}}% 
            \def\PYGZob{\discretionary{}{\Wrappedafterbreak\char`\{}{\char`\{}}% 
            \def\PYGZcb{\discretionary{\char`\}}{\Wrappedafterbreak}{\char`\}}}% 
            \def\PYGZca{\discretionary{\char`\^}{\Wrappedafterbreak}{\char`\^}}% 
            \def\PYGZam{\discretionary{\char`\&}{\Wrappedafterbreak}{\char`\&}}% 
            \def\PYGZlt{\discretionary{}{\Wrappedafterbreak\char`\<}{\char`\<}}% 
            \def\PYGZgt{\discretionary{\char`\>}{\Wrappedafterbreak}{\char`\>}}% 
            \def\PYGZsh{\discretionary{}{\Wrappedafterbreak\char`\#}{\char`\#}}% 
            \def\PYGZpc{\discretionary{}{\Wrappedafterbreak\char`\%}{\char`\%}}% 
            \def\PYGZdl{\discretionary{}{\Wrappedafterbreak\char`\$}{\char`\$}}% 
            \def\PYGZhy{\discretionary{\char`\-}{\Wrappedafterbreak}{\char`\-}}% 
            \def\PYGZsq{\discretionary{}{\Wrappedafterbreak\textquotesingle}{\textquotesingle}}% 
            \def\PYGZdq{\discretionary{}{\Wrappedafterbreak\char`\"}{\char`\"}}% 
            \def\PYGZti{\discretionary{\char`\~}{\Wrappedafterbreak}{\char`\~}}% 
        } 
        % Some characters . , ; ? ! / are not pygmentized. 
        % This macro makes them "active" and they will insert potential linebreaks 
        \newcommand*\Wrappedbreaksatpunct {% 
            \lccode`\~`\.\lowercase{\def~}{\discretionary{\hbox{\char`\.}}{\Wrappedafterbreak}{\hbox{\char`\.}}}% 
            \lccode`\~`\,\lowercase{\def~}{\discretionary{\hbox{\char`\,}}{\Wrappedafterbreak}{\hbox{\char`\,}}}% 
            \lccode`\~`\;\lowercase{\def~}{\discretionary{\hbox{\char`\;}}{\Wrappedafterbreak}{\hbox{\char`\;}}}% 
            \lccode`\~`\:\lowercase{\def~}{\discretionary{\hbox{\char`\:}}{\Wrappedafterbreak}{\hbox{\char`\:}}}% 
            \lccode`\~`\?\lowercase{\def~}{\discretionary{\hbox{\char`\?}}{\Wrappedafterbreak}{\hbox{\char`\?}}}% 
            \lccode`\~`\!\lowercase{\def~}{\discretionary{\hbox{\char`\!}}{\Wrappedafterbreak}{\hbox{\char`\!}}}% 
            \lccode`\~`\/\lowercase{\def~}{\discretionary{\hbox{\char`\/}}{\Wrappedafterbreak}{\hbox{\char`\/}}}% 
            \catcode`\.\active
            \catcode`\,\active 
            \catcode`\;\active
            \catcode`\:\active
            \catcode`\?\active
            \catcode`\!\active
            \catcode`\/\active 
            \lccode`\~`\~ 	
        }
    \makeatother

    \let\OriginalVerbatim=\Verbatim
    \makeatletter
    \renewcommand{\Verbatim}[1][1]{%
        %\parskip\z@skip
        \sbox\Wrappedcontinuationbox {\Wrappedcontinuationsymbol}%
        \sbox\Wrappedvisiblespacebox {\FV@SetupFont\Wrappedvisiblespace}%
        \def\FancyVerbFormatLine ##1{\hsize\linewidth
            \vtop{\raggedright\hyphenpenalty\z@\exhyphenpenalty\z@
                \doublehyphendemerits\z@\finalhyphendemerits\z@
                \strut ##1\strut}%
        }%
        % If the linebreak is at a space, the latter will be displayed as visible
        % space at end of first line, and a continuation symbol starts next line.
        % Stretch/shrink are however usually zero for typewriter font.
        \def\FV@Space {%
            \nobreak\hskip\z@ plus\fontdimen3\font minus\fontdimen4\font
            \discretionary{\copy\Wrappedvisiblespacebox}{\Wrappedafterbreak}
            {\kern\fontdimen2\font}%
        }%
        
        % Allow breaks at special characters using \PYG... macros.
        \Wrappedbreaksatspecials
        % Breaks at punctuation characters . , ; ? ! and / need catcode=\active 	
        \OriginalVerbatim[#1,fontsize=\small,codes*=\Wrappedbreaksatpunct]%
    }
    \makeatother

    % Exact colors from NB
    \definecolor{incolor}{HTML}{303F9F}
    \definecolor{outcolor}{HTML}{D84315}
    \definecolor{cellborder}{HTML}{CFCFCF}
    \definecolor{cellbackground}{HTML}{F7F7F7}
    
    % prompt
    \makeatletter
    \newcommand{\boxspacing}{\kern\kvtcb@left@rule\kern\kvtcb@boxsep}
    \makeatother
    \newcommand{\prompt}[4]{
        {\ttfamily\llap{{\color{#2}[#3]:\hspace{3pt}#4}}\vspace{-\baselineskip}}
    }
    

    
    % Prevent overflowing lines due to hard-to-break entities
    \sloppy 
    % Setup hyperref package
    \hypersetup{
      breaklinks=true,  % so long urls are correctly broken across lines
      colorlinks=true,
      urlcolor=urlcolor,
      linkcolor=linkcolor,
      citecolor=citecolor,
      }
    % Slightly bigger margins than the latex defaults
    
    \geometry{verbose,tmargin=1in,bmargin=1in,lmargin=1in,rmargin=1in}
    
    

\begin{document}
    
    \maketitle
    \tableofcontents

    

    
    \hypertarget{advanced-numpy}{%
\section{Advanced NumPy}\label{advanced-numpy}}

We already encountered NumPy arrays and their basic usage throughout
this tutorial. In this unit, we will take a more in-depth look at NumPy.

\hypertarget{why-numpy-arrays}{%
\subsection{Why NumPy arrays?}\label{why-numpy-arrays}}

Why don't we just stick with built-in types such as Python lists to
store and process data? It turns out that while the built-in objects are
quite flexible, this flexibility comes at the cost of decreased
performance:

\begin{itemize}
\item
  \texttt{list} objects can store arbitrary data types, and the data
  type of any item can change:

\begin{verbatim}
items = ['foo']
items[0] = 1.0      # item was a string, now it's a float!
\end{verbatim}
\item
  There is no guarantee where in memory the data will be stored. In
  fact, two consecutive items could be very ``far'' from each other in
  memory, which imposes a performance penalty.
\item
  Even primitive data types such as \texttt{int} and \texttt{float} are
  not ``raw'' data, but full-fledged objects. That, again, is bad for
  performance.
\end{itemize}

On the other hand, the approach taken by NumPy is to store and process
data in a way very similar to low-level languages such as C and Fortran.
This means that

\begin{itemize}
\item
  arrays contain a \emph{homogenous} data type. \emph{All} elements are
  either 64-bit integers (\texttt{np.int64}), 64-bit floating-point
  numbers (\texttt{np.float64}), or some other of the many data types
  supported by NumPy.

  It is technically possible to get around this by specifying an array's
  data type (\texttt{dtype}) to be \texttt{object}, which is the most
  generic Python data type. However, we would never want to do this for
  numerical computations.
\item
  NumPy arrays are usually \emph{contiguous} in memory. This means that
  adjacent array elements are actually guaranteed to be stored next to
  each other, which allows for much more efficient computations.
\item
  NumPy arrays support numerous operations used in scientific computing.
  For example, with a NumPy array we can write

\begin{verbatim}
x = np.array([1, 2, 3])
y = x + 1       # We would expect this to work
\end{verbatim}

  With lists, however, we cannot:

\begin{verbatim}
x = [1, 2, 3]
y = x + 1       # Does not work!
\end{verbatim}

  Lists don't implement an addition operator that accepts integer
  arguments, so this code triggers an error.
\end{itemize}

You can see the performance uplift provided by NumPy arrays in this
simple example:

    \begin{tcolorbox}[breakable, size=fbox, boxrule=1pt, pad at break*=1mm,colback=cellbackground, colframe=cellborder]
\prompt{In}{incolor}{1}{\boxspacing}
\begin{Verbatim}[commandchars=\\\{\}]
\PY{c+c1}{\PYZsh{} Create list 0, 1, 2, ..., 999}
\PY{n}{lst} \PY{o}{=} \PY{n+nb}{list}\PY{p}{(}\PY{n}{i} \PY{k}{for} \PY{n}{i} \PY{o+ow}{in} \PY{n+nb}{range}\PY{p}{(}\PY{l+m+mi}{1000}\PY{p}{)}\PY{p}{)}

\PY{c+c1}{\PYZsh{} Compute squares, time how long it takes}
\PY{o}{\PYZpc{}}\PY{k}{timeit} [i**2 for i in lst]
\end{Verbatim}
\end{tcolorbox}

    \begin{Verbatim}[commandchars=\\\{\}]
205 µs ± 12.3 µs per loop (mean ± std. dev. of 7 runs, 1000 loops each)
    \end{Verbatim}

    \begin{tcolorbox}[breakable, size=fbox, boxrule=1pt, pad at break*=1mm,colback=cellbackground, colframe=cellborder]
\prompt{In}{incolor}{2}{\boxspacing}
\begin{Verbatim}[commandchars=\\\{\}]
\PY{c+c1}{\PYZsh{} Repeat using NumPy arrays}
\PY{k+kn}{import} \PY{n+nn}{numpy} \PY{k}{as} \PY{n+nn}{np}
\PY{n}{arr} \PY{o}{=} \PY{n}{np}\PY{o}{.}\PY{n}{arange}\PY{p}{(}\PY{l+m+mi}{1000}\PY{p}{)}

\PY{o}{\PYZpc{}}\PY{k}{timeit} arr**2
\end{Verbatim}
\end{tcolorbox}

    \begin{Verbatim}[commandchars=\\\{\}]
1.15 µs ± 28 ns per loop (mean ± std. dev. of 7 runs, 1000000 loops each)
    \end{Verbatim}

    \begin{itemize}
\tightlist
\item
  On my machine (not the fastest laptop), squaring 1000 elements of a
  \texttt{list} takes approximately 200 microseconds, while the
  equivalent array-based operation takes only about 1 microsecond. NumPy
  is therefore approximately 200 times faster!
\item
  Also, as mentioned above, NumPy supports squaring an array directly,
  while we have to manually loop through the \texttt{list} and square
  each element individually.
\end{itemize}

\emph{Note:} \texttt{\%timeit} is a so-called magic command that only
works in notebooks, but not in regular Python files.
{[}\href{https://ipython.readthedocs.io/en/stable/interactive/magics.html}{See
documentation}{]}

    \begin{center}\rule{0.5\linewidth}{0.5pt}\end{center}

\hypertarget{creating-arrays}{%
\subsection{Creating arrays}\label{creating-arrays}}

We have already encountered some of the most frequently used array
creation routines:

\begin{itemize}
\tightlist
\item
  \texttt{np.array()} creates an array from a given argument, which can
  be

  \begin{itemize}
  \tightlist
  \item
    a scalar
  \item
    a collection such as a list or tuple
  \item
    some other iterable object, eg. something created by
    \texttt{range()}
  \end{itemize}
\item
  \texttt{np.empty()} allocates memory for a given array shape, but does
  not overwrite it with initial values.
\item
  \texttt{np.zeros()} creates an array of a given shape and initializes
  it to zeros.
\item
  \texttt{np.ones()} creates an array of a given shape and initializes
  it to ones.
\item
  \texttt{np.arange(start,stop,step)} creates an array with evenly
  spaced elements over the range \([start,stop)\).

  \begin{itemize}
  \tightlist
  \item
    \texttt{start} and \texttt{step} can be omitted and then default to
    \texttt{start=0} and \texttt{step=1}.
  \item
    Note that \texttt{stop} is not included!
  \end{itemize}
\item
  \texttt{np.linspace(start,stop,num)} returns a vector of \texttt{num}
  elements which are evenly spaced over the interval \([start,stop]\).
\item
  \texttt{np.identity(n)} returns the identity matrix of a size
  \(n \times n\).
\item
  \texttt{np.eye()} is a more flexible variant of \texttt{identity()}
  that can, for example, also create non-squared matrices.
\end{itemize}

There are many more array creation functions for more exotic use-cases,
see the NumPy
\href{https://numpy.org/doc/stable/reference/routines.array-creation.html}{documentation}
for details.

\emph{Examples:}

    \begin{tcolorbox}[breakable, size=fbox, boxrule=1pt, pad at break*=1mm,colback=cellbackground, colframe=cellborder]
\prompt{In}{incolor}{3}{\boxspacing}
\begin{Verbatim}[commandchars=\\\{\}]
\PY{k+kn}{import} \PY{n+nn}{numpy} \PY{k}{as} \PY{n+nn}{np}

\PY{c+c1}{\PYZsh{} Create array from list}
\PY{n}{lst} \PY{o}{=} \PY{p}{[}\PY{l+m+mi}{1}\PY{p}{,} \PY{l+m+mi}{2}\PY{p}{,} \PY{l+m+mi}{3}\PY{p}{]}
\PY{n}{np}\PY{o}{.}\PY{n}{array}\PY{p}{(}\PY{n}{lst}\PY{p}{)}
\end{Verbatim}
\end{tcolorbox}

            \begin{tcolorbox}[breakable, size=fbox, boxrule=.5pt, pad at break*=1mm, opacityfill=0]
\prompt{Out}{outcolor}{3}{\boxspacing}
\begin{Verbatim}[commandchars=\\\{\}]
array([1, 2, 3])
\end{Verbatim}
\end{tcolorbox}
        
    \begin{tcolorbox}[breakable, size=fbox, boxrule=1pt, pad at break*=1mm,colback=cellbackground, colframe=cellborder]
\prompt{In}{incolor}{4}{\boxspacing}
\begin{Verbatim}[commandchars=\\\{\}]
\PY{c+c1}{\PYZsh{} Create array from tuple}
\PY{n}{tpl} \PY{o}{=} \PY{l+m+mf}{1.0}\PY{p}{,} \PY{l+m+mf}{2.0}\PY{p}{,} \PY{l+m+mf}{3.0}
\PY{n}{np}\PY{o}{.}\PY{n}{array}\PY{p}{(}\PY{n}{tpl}\PY{p}{)}
\end{Verbatim}
\end{tcolorbox}

            \begin{tcolorbox}[breakable, size=fbox, boxrule=.5pt, pad at break*=1mm, opacityfill=0]
\prompt{Out}{outcolor}{4}{\boxspacing}
\begin{Verbatim}[commandchars=\\\{\}]
array([1., 2., 3.])
\end{Verbatim}
\end{tcolorbox}
        
    \begin{tcolorbox}[breakable, size=fbox, boxrule=1pt, pad at break*=1mm,colback=cellbackground, colframe=cellborder]
\prompt{In}{incolor}{5}{\boxspacing}
\begin{Verbatim}[commandchars=\\\{\}]
\PY{c+c1}{\PYZsh{} arange: end point is not included!}
\PY{n}{np}\PY{o}{.}\PY{n}{arange}\PY{p}{(}\PY{l+m+mi}{5}\PY{p}{)}
\end{Verbatim}
\end{tcolorbox}

            \begin{tcolorbox}[breakable, size=fbox, boxrule=.5pt, pad at break*=1mm, opacityfill=0]
\prompt{Out}{outcolor}{5}{\boxspacing}
\begin{Verbatim}[commandchars=\\\{\}]
array([0, 1, 2, 3, 4])
\end{Verbatim}
\end{tcolorbox}
        
    \begin{tcolorbox}[breakable, size=fbox, boxrule=1pt, pad at break*=1mm,colback=cellbackground, colframe=cellborder]
\prompt{In}{incolor}{6}{\boxspacing}
\begin{Verbatim}[commandchars=\\\{\}]
\PY{c+c1}{\PYZsh{} arange: increments can be negative too!}
\PY{n}{np}\PY{o}{.}\PY{n}{arange}\PY{p}{(}\PY{l+m+mi}{5}\PY{p}{,} \PY{l+m+mi}{1}\PY{p}{,} \PY{o}{\PYZhy{}}\PY{l+m+mi}{1}\PY{p}{)}
\end{Verbatim}
\end{tcolorbox}

            \begin{tcolorbox}[breakable, size=fbox, boxrule=.5pt, pad at break*=1mm, opacityfill=0]
\prompt{Out}{outcolor}{6}{\boxspacing}
\begin{Verbatim}[commandchars=\\\{\}]
array([5, 4, 3, 2])
\end{Verbatim}
\end{tcolorbox}
        
    \begin{tcolorbox}[breakable, size=fbox, boxrule=1pt, pad at break*=1mm,colback=cellbackground, colframe=cellborder]
\prompt{In}{incolor}{7}{\boxspacing}
\begin{Verbatim}[commandchars=\\\{\}]
\PY{c+c1}{\PYZsh{} arange also works on floats}
\PY{n}{np}\PY{o}{.}\PY{n}{arange}\PY{p}{(}\PY{l+m+mf}{1.0}\PY{p}{,} \PY{l+m+mf}{3.0}\PY{p}{,} \PY{l+m+mf}{0.5678}\PY{p}{)}
\end{Verbatim}
\end{tcolorbox}

            \begin{tcolorbox}[breakable, size=fbox, boxrule=.5pt, pad at break*=1mm, opacityfill=0]
\prompt{Out}{outcolor}{7}{\boxspacing}
\begin{Verbatim}[commandchars=\\\{\}]
array([1.    , 1.5678, 2.1356, 2.7034])
\end{Verbatim}
\end{tcolorbox}
        
    \begin{tcolorbox}[breakable, size=fbox, boxrule=1pt, pad at break*=1mm,colback=cellbackground, colframe=cellborder]
\prompt{In}{incolor}{8}{\boxspacing}
\begin{Verbatim}[commandchars=\\\{\}]
\PY{c+c1}{\PYZsh{} linspace DOES include the end point}
\PY{n}{np}\PY{o}{.}\PY{n}{linspace}\PY{p}{(}\PY{l+m+mf}{0.0}\PY{p}{,} \PY{l+m+mf}{1.0}\PY{p}{,} \PY{l+m+mi}{11}\PY{p}{)}
\end{Verbatim}
\end{tcolorbox}

            \begin{tcolorbox}[breakable, size=fbox, boxrule=.5pt, pad at break*=1mm, opacityfill=0]
\prompt{Out}{outcolor}{8}{\boxspacing}
\begin{Verbatim}[commandchars=\\\{\}]
array([0. , 0.1, 0.2, 0.3, 0.4, 0.5, 0.6, 0.7, 0.8, 0.9, 1. ])
\end{Verbatim}
\end{tcolorbox}
        
    \begin{center}\rule{0.5\linewidth}{0.5pt}\end{center}

\hypertarget{array-shape}{%
\subsection{Array shape}\label{array-shape}}

Many of the array creation routines take the desired shape of the array
as their first argument. Array shapes are usually specified as tuples:

\begin{itemize}
\item
  A vector with 5 elements has shape \texttt{(5,\ )}.

  Note the comma \texttt{,}: we need to specify a tuple with a single
  element using this comma, since \texttt{(5)} is just the integer 5,
  not a tuple.

  It is worth pointing out that this is not the same as a 2-dimensional
  array with shape \texttt{(1,\ 5)} or \texttt{(5,\ 1)}, even though
  they have the same number of elements.
\item
  A \(2\times2\) matrix has shape \texttt{(2,\ 2)}.
\item
  A higher-dimensional array has shape \texttt{(k,\ l,\ m,\ n,\ ...)}.
\item
  A \emph{scalar} NumPy array has shape \texttt{()}, an empty tuple.

  While ``scalar array'' sounds like an oxymoron, it does exist.
\end{itemize}

We can query the shape of an array using the \texttt{shape} attribute,
and the number of dimensions is stored in the \texttt{ndim} attribute.

\emph{Examples:}

    \begin{tcolorbox}[breakable, size=fbox, boxrule=1pt, pad at break*=1mm,colback=cellbackground, colframe=cellborder]
\prompt{In}{incolor}{9}{\boxspacing}
\begin{Verbatim}[commandchars=\\\{\}]
\PY{k+kn}{import} \PY{n+nn}{numpy} \PY{k}{as} \PY{n+nn}{np}

\PY{c+c1}{\PYZsh{} Scalar array}
\PY{n}{x} \PY{o}{=} \PY{n}{np}\PY{o}{.}\PY{n}{array}\PY{p}{(}\PY{l+m+mf}{0.0}\PY{p}{)}
\PY{n+nb}{print}\PY{p}{(}\PY{l+s+s1}{\PYZsq{}}\PY{l+s+s1}{Scalar array with shape=}\PY{l+s+si}{\PYZob{}\PYZcb{}}\PY{l+s+s1}{ and ndim=}\PY{l+s+si}{\PYZob{}\PYZcb{}}\PY{l+s+s1}{\PYZsq{}}\PY{o}{.}\PY{n}{format}\PY{p}{(}\PY{n}{x}\PY{o}{.}\PY{n}{shape}\PY{p}{,} \PY{n}{x}\PY{o}{.}\PY{n}{ndim}\PY{p}{)}\PY{p}{)}
\end{Verbatim}
\end{tcolorbox}

    \begin{Verbatim}[commandchars=\\\{\}]
Scalar array with shape=() and ndim=0
    \end{Verbatim}

    Note that a scalar NumPy array is not the same as a Python scalar. The
built-in type \texttt{float} has neither a \texttt{shape}, nor an
\texttt{ndim}, nor any other of the NumPy array attributes.

    \begin{tcolorbox}[breakable, size=fbox, boxrule=1pt, pad at break*=1mm,colback=cellbackground, colframe=cellborder]
\prompt{In}{incolor}{10}{\boxspacing}
\begin{Verbatim}[commandchars=\\\{\}]
\PY{n}{scalar} \PY{o}{=} \PY{l+m+mf}{1.0}
\PY{n}{scalar}\PY{o}{.}\PY{n}{shape}
\end{Verbatim}
\end{tcolorbox}

    \begin{Verbatim}[commandchars=\\\{\}, frame=single, framerule=2mm, rulecolor=\color{outerrorbackground}]
\textcolor{ansi-red}{---------------------------------------------------------------------------}
\textcolor{ansi-red}{AttributeError}                            Traceback (most recent call last)
\textcolor{ansi-green}{<ipython-input-1-cebbc052f34b>} in \textcolor{ansi-cyan}{<module>}
\textcolor{ansi-green-intense}{\textbf{      1}} scalar \textcolor{ansi-blue}{=} \textcolor{ansi-cyan}{1.0}
\textcolor{ansi-green}{----> 2}\textcolor{ansi-red}{ }scalar\textcolor{ansi-blue}{.}shape

\textcolor{ansi-red}{AttributeError}: 'float' object has no attribute 'shape'
    \end{Verbatim}

    \begin{tcolorbox}[breakable, size=fbox, boxrule=1pt, pad at break*=1mm,colback=cellbackground, colframe=cellborder]
\prompt{In}{incolor}{11}{\boxspacing}
\begin{Verbatim}[commandchars=\\\{\}]
\PY{c+c1}{\PYZsh{} 1\PYZhy{}dimensional array (vector), values not initialised}
\PY{n}{x} \PY{o}{=} \PY{n}{np}\PY{o}{.}\PY{n}{empty}\PY{p}{(}\PY{p}{(}\PY{l+m+mi}{5}\PY{p}{,}\PY{p}{)}\PY{p}{)}
\PY{n}{x}       \PY{c+c1}{\PYZsh{} could contain arbitrary garbage}
\end{Verbatim}
\end{tcolorbox}

            \begin{tcolorbox}[breakable, size=fbox, boxrule=.5pt, pad at break*=1mm, opacityfill=0]
\prompt{Out}{outcolor}{11}{\boxspacing}
\begin{Verbatim}[commandchars=\\\{\}]
array([0.0e+000, 4.9e-324, 9.9e-324, 1.5e-323, 2.0e-323])
\end{Verbatim}
\end{tcolorbox}
        
    An array created with \texttt{empty()} will contain arbitrary garbage
since the memory block assigned to the array is not initialised. The
result will potentially differ on each invocation and across computers.

Most function accept an integer value instead of a \texttt{tuple} when
creating 1-dimensional arrays, which is interpreted as the number of
elements:

    \begin{tcolorbox}[breakable, size=fbox, boxrule=1pt, pad at break*=1mm,colback=cellbackground, colframe=cellborder]
\prompt{In}{incolor}{12}{\boxspacing}
\begin{Verbatim}[commandchars=\\\{\}]
\PY{c+c1}{\PYZsh{} 1\PYZhy{}dimensional array}
\PY{n}{x} \PY{o}{=} \PY{n}{np}\PY{o}{.}\PY{n}{empty}\PY{p}{(}\PY{l+m+mi}{5}\PY{p}{)}         \PY{c+c1}{\PYZsh{} equivalent to np.empty((5,))}
\end{Verbatim}
\end{tcolorbox}

    Higher-dimensional arrays are creating by passing in tuples with more
than one element:

    \begin{tcolorbox}[breakable, size=fbox, boxrule=1pt, pad at break*=1mm,colback=cellbackground, colframe=cellborder]
\prompt{In}{incolor}{13}{\boxspacing}
\begin{Verbatim}[commandchars=\\\{\}]
\PY{n}{np}\PY{o}{.}\PY{n}{ones}\PY{p}{(}\PY{p}{(}\PY{l+m+mi}{1}\PY{p}{,} \PY{l+m+mi}{2}\PY{p}{,} \PY{l+m+mi}{3}\PY{p}{)}\PY{p}{)}      \PY{c+c1}{\PYZsh{} 3d\PYZhy{}array}
\end{Verbatim}
\end{tcolorbox}

            \begin{tcolorbox}[breakable, size=fbox, boxrule=.5pt, pad at break*=1mm, opacityfill=0]
\prompt{Out}{outcolor}{13}{\boxspacing}
\begin{Verbatim}[commandchars=\\\{\}]
array([[[1., 1., 1.],
        [1., 1., 1.]]])
\end{Verbatim}
\end{tcolorbox}
        
    Recall from unit 2 that we can use the \texttt{reshape()} method to
convert arrays to a different shape:

\begin{itemize}
\tightlist
\item
  The resulting number of elements must remain unchanged!
\item
  \emph{One} dimension can be specified using \texttt{-1}, which will
  prompt NumPy to compute the implied dimension size itself.
\end{itemize}

    \begin{tcolorbox}[breakable, size=fbox, boxrule=1pt, pad at break*=1mm,colback=cellbackground, colframe=cellborder]
\prompt{In}{incolor}{14}{\boxspacing}
\begin{Verbatim}[commandchars=\\\{\}]
\PY{n}{x} \PY{o}{=} \PY{n}{np}\PY{o}{.}\PY{n}{zeros}\PY{p}{(}\PY{p}{(}\PY{l+m+mi}{2}\PY{p}{,} \PY{l+m+mi}{1}\PY{p}{,} \PY{l+m+mi}{3}\PY{p}{)}\PY{p}{)}
\PY{n}{x} \PY{o}{=} \PY{n}{x}\PY{o}{.}\PY{n}{reshape}\PY{p}{(}\PY{p}{(}\PY{l+m+mi}{3}\PY{p}{,} \PY{o}{\PYZhy{}}\PY{l+m+mi}{1}\PY{p}{)}\PY{p}{)}      \PY{c+c1}{\PYZsh{} Infer number of columns}
\PY{n}{x}
\end{Verbatim}
\end{tcolorbox}

            \begin{tcolorbox}[breakable, size=fbox, boxrule=.5pt, pad at break*=1mm, opacityfill=0]
\prompt{Out}{outcolor}{14}{\boxspacing}
\begin{Verbatim}[commandchars=\\\{\}]
array([[0., 0.],
       [0., 0.],
       [0., 0.]])
\end{Verbatim}
\end{tcolorbox}
        
    \begin{center}\rule{0.5\linewidth}{0.5pt}\end{center}

\hypertarget{advanced-indexing}{%
\subsection{Advanced indexing}\label{advanced-indexing}}

We previously covered single element indexing and slicing, which works
the same way for both Python \texttt{list} and \texttt{tuple} objects as
well as NumPy arrays.

NumPy additionally implements more sophisticated indexing mechanisms
which we cover now.

\begin{itemize}
\tightlist
\item
  You might also want to consult the NumPy indexing
  \href{https://numpy.org/doc/stable/user/basics.indexing.html}{tutorial}
  and the detailed indexing
  \href{https://numpy.org/doc/stable/reference/arrays.indexing.html}{reference}.
\end{itemize}

\hypertarget{boolean-or-mask-indexing}{%
\subsubsection{Boolean or ``mask''
indexing}\label{boolean-or-mask-indexing}}

We can pass logical arrays as indices:

\begin{itemize}
\tightlist
\item
  Logical (or boolean) arrays consist of elements that can only take on
  values \texttt{True} and \texttt{False}
\item
  We usually don't create logical arrays manually, but apply an
  operation that results in \texttt{True}/\texttt{False} values, such as
  a comparison.
\item
  The boolean index array usually has the \emph{same} shape as the
  indexed array.
\end{itemize}

\emph{Examples:}

    \begin{tcolorbox}[breakable, size=fbox, boxrule=1pt, pad at break*=1mm,colback=cellbackground, colframe=cellborder]
\prompt{In}{incolor}{15}{\boxspacing}
\begin{Verbatim}[commandchars=\\\{\}]
\PY{k+kn}{import} \PY{n+nn}{numpy} \PY{k}{as} \PY{n+nn}{np}

\PY{n}{vec} \PY{o}{=} \PY{n}{np}\PY{o}{.}\PY{n}{arange}\PY{p}{(}\PY{l+m+mi}{5}\PY{p}{)}
\PY{n}{mask} \PY{o}{=} \PY{p}{(}\PY{n}{vec} \PY{o}{\PYZgt{}} \PY{l+m+mi}{1}\PY{p}{)}        \PY{c+c1}{\PYZsh{} apply comparison to create boolean array}
\PY{n}{mask}
\end{Verbatim}
\end{tcolorbox}

            \begin{tcolorbox}[breakable, size=fbox, boxrule=.5pt, pad at break*=1mm, opacityfill=0]
\prompt{Out}{outcolor}{15}{\boxspacing}
\begin{Verbatim}[commandchars=\\\{\}]
array([False, False,  True,  True,  True])
\end{Verbatim}
\end{tcolorbox}
        
    \begin{tcolorbox}[breakable, size=fbox, boxrule=1pt, pad at break*=1mm,colback=cellbackground, colframe=cellborder]
\prompt{In}{incolor}{16}{\boxspacing}
\begin{Verbatim}[commandchars=\\\{\}]
\PY{n}{vec}\PY{p}{[}\PY{n}{mask}\PY{p}{]}               \PY{c+c1}{\PYZsh{} use mask to retrieve only elements greater than 1}
\end{Verbatim}
\end{tcolorbox}

            \begin{tcolorbox}[breakable, size=fbox, boxrule=.5pt, pad at break*=1mm, opacityfill=0]
\prompt{Out}{outcolor}{16}{\boxspacing}
\begin{Verbatim}[commandchars=\\\{\}]
array([2, 3, 4])
\end{Verbatim}
\end{tcolorbox}
        
    We can even apply boolean indexing to multi-dimensional arrays. The
result will be flatted to a 1-dimensional array, though.

    \begin{tcolorbox}[breakable, size=fbox, boxrule=1pt, pad at break*=1mm,colback=cellbackground, colframe=cellborder]
\prompt{In}{incolor}{17}{\boxspacing}
\begin{Verbatim}[commandchars=\\\{\}]
\PY{n}{mat} \PY{o}{=} \PY{n}{np}\PY{o}{.}\PY{n}{arange}\PY{p}{(}\PY{l+m+mi}{6}\PY{p}{)}\PY{o}{.}\PY{n}{reshape}\PY{p}{(}\PY{p}{(}\PY{l+m+mi}{2}\PY{p}{,}\PY{l+m+mi}{3}\PY{p}{)}\PY{p}{)}
\PY{n}{mat}
\end{Verbatim}
\end{tcolorbox}

            \begin{tcolorbox}[breakable, size=fbox, boxrule=.5pt, pad at break*=1mm, opacityfill=0]
\prompt{Out}{outcolor}{17}{\boxspacing}
\begin{Verbatim}[commandchars=\\\{\}]
array([[0, 1, 2],
       [3, 4, 5]])
\end{Verbatim}
\end{tcolorbox}
        
    \begin{tcolorbox}[breakable, size=fbox, boxrule=1pt, pad at break*=1mm,colback=cellbackground, colframe=cellborder]
\prompt{In}{incolor}{18}{\boxspacing}
\begin{Verbatim}[commandchars=\\\{\}]
\PY{n}{mask} \PY{o}{=} \PY{p}{(}\PY{n}{mat} \PY{o}{\PYZgt{}} \PY{l+m+mi}{1}\PY{p}{)}        \PY{c+c1}{\PYZsh{} create boolean array}
\PY{n}{mask}
\end{Verbatim}
\end{tcolorbox}

            \begin{tcolorbox}[breakable, size=fbox, boxrule=.5pt, pad at break*=1mm, opacityfill=0]
\prompt{Out}{outcolor}{18}{\boxspacing}
\begin{Verbatim}[commandchars=\\\{\}]
array([[False, False,  True],
       [ True,  True,  True]])
\end{Verbatim}
\end{tcolorbox}
        
    \begin{tcolorbox}[breakable, size=fbox, boxrule=1pt, pad at break*=1mm,colback=cellbackground, colframe=cellborder]
\prompt{In}{incolor}{19}{\boxspacing}
\begin{Verbatim}[commandchars=\\\{\}]
\PY{n}{mat}\PY{p}{[}\PY{n}{mask}\PY{p}{]}            \PY{c+c1}{\PYZsh{} collapses result to 1\PYZhy{}d array}
\end{Verbatim}
\end{tcolorbox}

            \begin{tcolorbox}[breakable, size=fbox, boxrule=.5pt, pad at break*=1mm, opacityfill=0]
\prompt{Out}{outcolor}{19}{\boxspacing}
\begin{Verbatim}[commandchars=\\\{\}]
array([2, 3, 4, 5])
\end{Verbatim}
\end{tcolorbox}
        
    Logical indexing does \textbf{not} work with \texttt{tuple} and
\texttt{list}

    \begin{tcolorbox}[breakable, size=fbox, boxrule=1pt, pad at break*=1mm,colback=cellbackground, colframe=cellborder]
\prompt{In}{incolor}{20}{\boxspacing}
\begin{Verbatim}[commandchars=\\\{\}]
\PY{n}{tpl} \PY{o}{=} \PY{p}{(}\PY{l+m+mi}{1}\PY{p}{,} \PY{l+m+mi}{2}\PY{p}{,} \PY{l+m+mi}{3}\PY{p}{)}
\PY{n}{mask} \PY{o}{=} \PY{p}{(}\PY{k+kc}{True}\PY{p}{,} \PY{k+kc}{False}\PY{p}{,} \PY{k+kc}{True}\PY{p}{)}
\PY{n}{tpl}\PY{p}{[}\PY{n}{mask}\PY{p}{]}               \PY{c+c1}{\PYZsh{} error}
\end{Verbatim}
\end{tcolorbox}

    \begin{Verbatim}[commandchars=\\\{\}, frame=single, framerule=2mm, rulecolor=\color{outerrorbackground}]
\textcolor{ansi-red}{---------------------------------------------------------------------------}
\textcolor{ansi-red}{TypeError}                                 Traceback (most recent call last)
\textcolor{ansi-green}{<ipython-input-1-19bb25c629fe>} in \textcolor{ansi-cyan}{<module>}
\textcolor{ansi-green-intense}{\textbf{      1}} tpl \textcolor{ansi-blue}{=} \textcolor{ansi-blue}{(}\textcolor{ansi-cyan}{1}\textcolor{ansi-blue}{,} \textcolor{ansi-cyan}{2}\textcolor{ansi-blue}{,} \textcolor{ansi-cyan}{3}\textcolor{ansi-blue}{)}
\textcolor{ansi-green-intense}{\textbf{      2}} mask \textcolor{ansi-blue}{=} \textcolor{ansi-blue}{(}\textcolor{ansi-green}{True}\textcolor{ansi-blue}{,} \textcolor{ansi-green}{False}\textcolor{ansi-blue}{,} \textcolor{ansi-green}{True}\textcolor{ansi-blue}{)}
\textcolor{ansi-green}{----> 3}\textcolor{ansi-red}{ }tpl\textcolor{ansi-blue}{[}mask\textcolor{ansi-blue}{]}               \textcolor{ansi-red}{\# error}

\textcolor{ansi-red}{TypeError}: tuple indices must be integers or slices, not tuple
    \end{Verbatim}

    \hypertarget{integer-index-arrays}{%
\subsubsection{Integer index arrays}\label{integer-index-arrays}}

We can also use index arrays of \emph{integer} type to select specific
elements on each axis. These are straightforward to use for
1-dimensional arrays, but can get fairly complex with multiple
dimensions.

    \begin{tcolorbox}[breakable, size=fbox, boxrule=1pt, pad at break*=1mm,colback=cellbackground, colframe=cellborder]
\prompt{In}{incolor}{21}{\boxspacing}
\begin{Verbatim}[commandchars=\\\{\}]
\PY{k+kn}{import} \PY{n+nn}{numpy} \PY{k}{as} \PY{n+nn}{np}

\PY{n}{data} \PY{o}{=} \PY{n}{np}\PY{o}{.}\PY{n}{arange}\PY{p}{(}\PY{l+m+mi}{10}\PY{p}{)}
\PY{n}{index} \PY{o}{=} \PY{p}{[}\PY{l+m+mi}{1}\PY{p}{,} \PY{l+m+mi}{2}\PY{p}{,} \PY{l+m+mi}{9}\PY{p}{]}       \PY{c+c1}{\PYZsh{} select second, third and 10th element}
\PY{n}{data}\PY{p}{[}\PY{n}{index}\PY{p}{]}
\end{Verbatim}
\end{tcolorbox}

            \begin{tcolorbox}[breakable, size=fbox, boxrule=.5pt, pad at break*=1mm, opacityfill=0]
\prompt{Out}{outcolor}{21}{\boxspacing}
\begin{Verbatim}[commandchars=\\\{\}]
array([1, 2, 9])
\end{Verbatim}
\end{tcolorbox}
        
    As you see, the index array does not have to be a NumPy array, but can
also be a list (not a tuple, though!).

In general, if we are using an index array to select elements along an
axis of length \(n\), then

\begin{itemize}
\tightlist
\item
  the index must only contain integers between \(0\) and \(n-1\), or
  negative integers from \(-n\) to \(-1\) (which, as usual, count from
  the end of the axis).
\item
  the index can be of arbitrary length. We can therefore select the same
  element multiple times.
\end{itemize}

    \begin{tcolorbox}[breakable, size=fbox, boxrule=1pt, pad at break*=1mm,colback=cellbackground, colframe=cellborder]
\prompt{In}{incolor}{22}{\boxspacing}
\begin{Verbatim}[commandchars=\\\{\}]
\PY{n}{data} \PY{o}{=} \PY{n}{np}\PY{o}{.}\PY{n}{arange}\PY{p}{(}\PY{l+m+mi}{5}\PY{p}{,} \PY{l+m+mi}{10}\PY{p}{)}     \PY{c+c1}{\PYZsh{} array with 5 elements, [5,...,9]}
\PY{n}{data}
\end{Verbatim}
\end{tcolorbox}

            \begin{tcolorbox}[breakable, size=fbox, boxrule=.5pt, pad at break*=1mm, opacityfill=0]
\prompt{Out}{outcolor}{22}{\boxspacing}
\begin{Verbatim}[commandchars=\\\{\}]
array([5, 6, 7, 8, 9])
\end{Verbatim}
\end{tcolorbox}
        
    \begin{tcolorbox}[breakable, size=fbox, boxrule=1pt, pad at break*=1mm,colback=cellbackground, colframe=cellborder]
\prompt{In}{incolor}{23}{\boxspacing}
\begin{Verbatim}[commandchars=\\\{\}]
\PY{n}{index} \PY{o}{=} \PY{p}{[}\PY{l+m+mi}{0}\PY{p}{,} \PY{l+m+mi}{1}\PY{p}{,} \PY{l+m+mi}{1}\PY{p}{,} \PY{l+m+mi}{2}\PY{p}{,} \PY{l+m+mi}{2}\PY{p}{,} \PY{l+m+mi}{3}\PY{p}{,} \PY{l+m+mi}{3}\PY{p}{,} \PY{l+m+mi}{4}\PY{p}{,} \PY{l+m+mi}{4}\PY{p}{]}         \PY{c+c1}{\PYZsh{} select elements multiple times}
\PY{n}{data}\PY{p}{[}\PY{n}{index}\PY{p}{]}
\end{Verbatim}
\end{tcolorbox}

            \begin{tcolorbox}[breakable, size=fbox, boxrule=.5pt, pad at break*=1mm, opacityfill=0]
\prompt{Out}{outcolor}{23}{\boxspacing}
\begin{Verbatim}[commandchars=\\\{\}]
array([5, 6, 6, 7, 7, 8, 8, 9, 9])
\end{Verbatim}
\end{tcolorbox}
        
    With multi-dimensional arrays, the same restrictions apply. Moreover,

\begin{itemize}
\tightlist
\item
  if more than one axis is indexed using index arrays, the index arrays
  have to be of equal length.
\item
  we can combine integer array indexing on one axis with other types of
  indices on the remaining axes.
\end{itemize}

    \begin{tcolorbox}[breakable, size=fbox, boxrule=1pt, pad at break*=1mm,colback=cellbackground, colframe=cellborder]
\prompt{In}{incolor}{24}{\boxspacing}
\begin{Verbatim}[commandchars=\\\{\}]
\PY{n}{data} \PY{o}{=} \PY{n}{np}\PY{o}{.}\PY{n}{arange}\PY{p}{(}\PY{l+m+mi}{12}\PY{p}{)}\PY{o}{.}\PY{n}{reshape}\PY{p}{(}\PY{p}{(}\PY{l+m+mi}{3}\PY{p}{,} \PY{l+m+mi}{4}\PY{p}{)}\PY{p}{)}
\PY{n}{data}
\end{Verbatim}
\end{tcolorbox}

            \begin{tcolorbox}[breakable, size=fbox, boxrule=.5pt, pad at break*=1mm, opacityfill=0]
\prompt{Out}{outcolor}{24}{\boxspacing}
\begin{Verbatim}[commandchars=\\\{\}]
array([[ 0,  1,  2,  3],
       [ 4,  5,  6,  7],
       [ 8,  9, 10, 11]])
\end{Verbatim}
\end{tcolorbox}
        
    \begin{tcolorbox}[breakable, size=fbox, boxrule=1pt, pad at break*=1mm,colback=cellbackground, colframe=cellborder]
\prompt{In}{incolor}{25}{\boxspacing}
\begin{Verbatim}[commandchars=\\\{\}]
\PY{n}{index1} \PY{o}{=} \PY{p}{[}\PY{l+m+mi}{0}\PY{p}{,} \PY{l+m+mi}{2}\PY{p}{]}     \PY{c+c1}{\PYZsh{} row indices}
\PY{n}{index2} \PY{o}{=} \PY{p}{[}\PY{l+m+mi}{1}\PY{p}{,} \PY{l+m+mi}{3}\PY{p}{]}     \PY{c+c1}{\PYZsh{} column indices}
\PY{n}{data}\PY{p}{[}\PY{n}{index1}\PY{p}{,} \PY{n}{index2}\PY{p}{]}
\end{Verbatim}
\end{tcolorbox}

            \begin{tcolorbox}[breakable, size=fbox, boxrule=.5pt, pad at break*=1mm, opacityfill=0]
\prompt{Out}{outcolor}{25}{\boxspacing}
\begin{Verbatim}[commandchars=\\\{\}]
array([ 1, 11])
\end{Verbatim}
\end{tcolorbox}
        
    The code above selects to elements, the first at position
\texttt{(0,1)}, the second at position \texttt{(2,3)}.

We can combine index arrays on one axis with other indexing methods used
for another axis:

    \begin{tcolorbox}[breakable, size=fbox, boxrule=1pt, pad at break*=1mm,colback=cellbackground, colframe=cellborder]
\prompt{In}{incolor}{26}{\boxspacing}
\begin{Verbatim}[commandchars=\\\{\}]
\PY{n}{data}\PY{p}{[}\PY{n}{index1}\PY{p}{,} \PY{l+m+mi}{2}\PY{p}{]}     \PY{c+c1}{\PYZsh{} return elements in 3rd column from rows given in index1}
\end{Verbatim}
\end{tcolorbox}

            \begin{tcolorbox}[breakable, size=fbox, boxrule=.5pt, pad at break*=1mm, opacityfill=0]
\prompt{Out}{outcolor}{26}{\boxspacing}
\begin{Verbatim}[commandchars=\\\{\}]
array([ 2, 10])
\end{Verbatim}
\end{tcolorbox}
        
    Using different indexing methods, in particular index arrays, on
higher-dimensional data can quickly become a mess, and you should be
extra careful to see if the results make sense.

    \begin{center}\rule{0.5\linewidth}{0.5pt}\end{center}

\hypertarget{numerical-operations}{%
\subsection{Numerical operations}\label{numerical-operations}}

\hypertarget{element-wise-operations}{%
\subsubsection{Element-wise operations}\label{element-wise-operations}}

Element-wise operations are performed on each element individually and
leave the resulting array's shape unchanged.

There are three types of such operations:

\begin{enumerate}
\def\labelenumi{\arabic{enumi}.}
\tightlist
\item
  One operand is an array and one is a scalar.
\item
  Both operands are arrays, either of identical shape, or broadcastable
  to an identical shape (we discuss broadcasting below)
\item
  A function is applied to each array element.
\end{enumerate}

\emph{Case 1:} Array-scalar operations. These intuitively behave as you
would expect:

    \begin{tcolorbox}[breakable, size=fbox, boxrule=1pt, pad at break*=1mm,colback=cellbackground, colframe=cellborder]
\prompt{In}{incolor}{27}{\boxspacing}
\begin{Verbatim}[commandchars=\\\{\}]
\PY{k+kn}{import} \PY{n+nn}{numpy} \PY{k}{as} \PY{n+nn}{np}

\PY{n}{x} \PY{o}{=} \PY{n}{np}\PY{o}{.}\PY{n}{arange}\PY{p}{(}\PY{l+m+mi}{10}\PY{p}{)}
\PY{n}{scalar} \PY{o}{=} \PY{l+m+mi}{1}

\PY{c+c1}{\PYZsh{} The resulting array y has the same shape as x:}
\PY{n}{y} \PY{o}{=} \PY{n}{x} \PY{o}{+} \PY{n}{scalar}      \PY{c+c1}{\PYZsh{} addition}
\PY{n}{y} \PY{o}{=} \PY{n}{x} \PY{o}{\PYZhy{}} \PY{n}{scalar}      \PY{c+c1}{\PYZsh{} subtraction}
\PY{n}{y} \PY{o}{=} \PY{n}{x} \PY{o}{*} \PY{n}{scalar}      \PY{c+c1}{\PYZsh{} multiplication}
\PY{n}{y} \PY{o}{=} \PY{n}{x} \PY{o}{/} \PY{n}{scalar}      \PY{c+c1}{\PYZsh{} division}
\PY{n}{y} \PY{o}{=} \PY{n}{x} \PY{o}{/}\PY{o}{/} \PY{n}{scalar}     \PY{c+c1}{\PYZsh{} division with integer truncation}
\PY{n}{y} \PY{o}{=} \PY{n}{x} \PY{o}{\PYZpc{}} \PY{n}{scalar}      \PY{c+c1}{\PYZsh{} modulo operator}
\PY{n}{y} \PY{o}{=} \PY{n}{x} \PY{o}{*}\PY{o}{*} \PY{n}{scalar}     \PY{c+c1}{\PYZsh{} power function}
\PY{n}{y} \PY{o}{=} \PY{n}{x} \PY{o}{==} \PY{n}{scalar}     \PY{c+c1}{\PYZsh{} comparison: also \PYZgt{}, \PYZgt{}=, \PYZlt{}=, \PYZlt{}}
\end{Verbatim}
\end{tcolorbox}

    Note that unlike in Matlab, the ``standard'' operators work
element-wise, so \texttt{x\ *\ y} is \textbf{not} matrix multiplication!

\emph{Case 2:} Both operands are arrays of equal shape:

    \begin{tcolorbox}[breakable, size=fbox, boxrule=1pt, pad at break*=1mm,colback=cellbackground, colframe=cellborder]
\prompt{In}{incolor}{28}{\boxspacing}
\begin{Verbatim}[commandchars=\\\{\}]
\PY{n}{x} \PY{o}{=} \PY{n}{np}\PY{o}{.}\PY{n}{arange}\PY{p}{(}\PY{l+m+mi}{10}\PY{p}{)}
\PY{n}{y} \PY{o}{=} \PY{n}{np}\PY{o}{.}\PY{n}{arange}\PY{p}{(}\PY{l+m+mi}{10}\PY{p}{,} \PY{l+m+mi}{20}\PY{p}{)}       \PY{c+c1}{\PYZsh{} has same shape as x}

\PY{c+c1}{\PYZsh{} Resulting array z has the same shape as x and y:}
\PY{n}{z} \PY{o}{=} \PY{n}{x} \PY{o}{+} \PY{n}{y}           \PY{c+c1}{\PYZsh{} addition}
\PY{n}{z} \PY{o}{=} \PY{n}{x} \PY{o}{\PYZhy{}} \PY{n}{y}           \PY{c+c1}{\PYZsh{} subtraction}
\PY{n}{z} \PY{o}{=} \PY{n}{x} \PY{o}{*} \PY{n}{y}           \PY{c+c1}{\PYZsh{} multiplication}
\PY{n}{z} \PY{o}{=} \PY{n}{x} \PY{o}{/} \PY{n}{y}           \PY{c+c1}{\PYZsh{} division}
\PY{n}{z} \PY{o}{=} \PY{n}{x} \PY{o}{/}\PY{o}{/} \PY{n}{y}          \PY{c+c1}{\PYZsh{} division with integer truncation}
\PY{n}{z} \PY{o}{=} \PY{n}{x} \PY{o}{\PYZpc{}} \PY{n}{y}           \PY{c+c1}{\PYZsh{} modulo operator}
\PY{n}{z} \PY{o}{=} \PY{n}{x} \PY{o}{*}\PY{o}{*} \PY{n}{y}          \PY{c+c1}{\PYZsh{} power function}
\PY{n}{y} \PY{o}{=} \PY{n}{x} \PY{o}{==} \PY{n}{y}          \PY{c+c1}{\PYZsh{} comparison: also \PYZgt{}, \PYZgt{}=, \PYZlt{}=, \PYZlt{}}
\end{Verbatim}
\end{tcolorbox}

    \emph{Case 3:} Applying element-wise functions. This case covers
numerous functions defined in NumPy, such as

\begin{itemize}
\tightlist
\item
  \texttt{np.sqrt}: square root
\item
  \texttt{np.exp}, \texttt{np.log}, \texttt{np.log10}: exponential and
  logarithmic functions
\item
  \texttt{np.sin}, \texttt{np.cos}, etc.: trigonometric functions
\end{itemize}

    You can find a complete list of mathematical functions in the NumPy
\href{https://numpy.org/doc/stable/reference/routines.math.html}{documentation}
(not all functions listed there operate element-wise!).

    \begin{tcolorbox}[breakable, size=fbox, boxrule=1pt, pad at break*=1mm,colback=cellbackground, colframe=cellborder]
\prompt{In}{incolor}{29}{\boxspacing}
\begin{Verbatim}[commandchars=\\\{\}]
\PY{c+c1}{\PYZsh{} element\PYZhy{}wise functions}
\PY{n}{x} \PY{o}{=} \PY{n}{np}\PY{o}{.}\PY{n}{arange}\PY{p}{(}\PY{l+m+mi}{1}\PY{p}{,} \PY{l+m+mi}{11}\PY{p}{)}
\PY{n}{y} \PY{o}{=} \PY{n}{np}\PY{o}{.}\PY{n}{exp}\PY{p}{(}\PY{n}{x}\PY{p}{)}       \PY{c+c1}{\PYZsh{} apply exponential function}
\PY{n}{y} \PY{o}{=} \PY{n}{np}\PY{o}{.}\PY{n}{log}\PY{p}{(}\PY{n}{x}\PY{p}{)}       \PY{c+c1}{\PYZsh{} apply natural logarithm}
\end{Verbatim}
\end{tcolorbox}

    \hypertarget{matrix-operations}{%
\subsubsection{Matrix operations}\label{matrix-operations}}

\textbf{Transpose}

You can transpose a matrix using the \texttt{T} attribute:

    \begin{tcolorbox}[breakable, size=fbox, boxrule=1pt, pad at break*=1mm,colback=cellbackground, colframe=cellborder]
\prompt{In}{incolor}{30}{\boxspacing}
\begin{Verbatim}[commandchars=\\\{\}]
\PY{n}{mat} \PY{o}{=} \PY{n}{np}\PY{o}{.}\PY{n}{arange}\PY{p}{(}\PY{l+m+mi}{6}\PY{p}{)}\PY{o}{.}\PY{n}{reshape}\PY{p}{(}\PY{p}{(}\PY{l+m+mi}{2}\PY{p}{,}\PY{l+m+mi}{3}\PY{p}{)}\PY{p}{)}
\PY{n}{mat}
\end{Verbatim}
\end{tcolorbox}

            \begin{tcolorbox}[breakable, size=fbox, boxrule=.5pt, pad at break*=1mm, opacityfill=0]
\prompt{Out}{outcolor}{30}{\boxspacing}
\begin{Verbatim}[commandchars=\\\{\}]
array([[0, 1, 2],
       [3, 4, 5]])
\end{Verbatim}
\end{tcolorbox}
        
    \begin{tcolorbox}[breakable, size=fbox, boxrule=1pt, pad at break*=1mm,colback=cellbackground, colframe=cellborder]
\prompt{In}{incolor}{31}{\boxspacing}
\begin{Verbatim}[commandchars=\\\{\}]
\PY{n}{mat}\PY{o}{.}\PY{n}{T}
\end{Verbatim}
\end{tcolorbox}

            \begin{tcolorbox}[breakable, size=fbox, boxrule=.5pt, pad at break*=1mm, opacityfill=0]
\prompt{Out}{outcolor}{31}{\boxspacing}
\begin{Verbatim}[commandchars=\\\{\}]
array([[0, 3],
       [1, 4],
       [2, 5]])
\end{Verbatim}
\end{tcolorbox}
        
    For higher-dimensional arrays, the \texttt{np.transpose()} function can
be used to permute the axes of an array. For two-dimensional arrays,
\texttt{np.transpose(mat)} and \texttt{mat.T} are equivalent.

\textbf{Array multiplication}

Matrix multiplication is performed using the \texttt{np.dot()} function
(``dot product''). The operands need not be matrices but can be vectors
as well, or even high-dimensional arrays (the result is then not
entirely obvious and one should check the
\href{https://numpy.org/doc/stable/reference/generated/numpy.dot.html}{documentation}).

Every newer version of Python and NumPy additionally interprets
\texttt{@} as the matrix multiplication operator.

    \begin{tcolorbox}[breakable, size=fbox, boxrule=1pt, pad at break*=1mm,colback=cellbackground, colframe=cellborder]
\prompt{In}{incolor}{32}{\boxspacing}
\begin{Verbatim}[commandchars=\\\{\}]
\PY{k+kn}{import} \PY{n+nn}{numpy} \PY{k}{as} \PY{n+nn}{np}

\PY{n}{mat} \PY{o}{=} \PY{n}{np}\PY{o}{.}\PY{n}{arange}\PY{p}{(}\PY{l+m+mi}{9}\PY{p}{)}\PY{o}{.}\PY{n}{reshape}\PY{p}{(}\PY{p}{(}\PY{l+m+mi}{3}\PY{p}{,} \PY{l+m+mi}{3}\PY{p}{)}\PY{p}{)}
\PY{n}{vec} \PY{o}{=} \PY{n}{np}\PY{o}{.}\PY{n}{arange}\PY{p}{(}\PY{l+m+mi}{3}\PY{p}{)}

\PY{c+c1}{\PYZsh{} matrix\PYZhy{}matrix multiplication}
\PY{n}{np}\PY{o}{.}\PY{n}{dot}\PY{p}{(}\PY{n}{mat}\PY{p}{,} \PY{n}{mat}\PY{p}{)}    \PY{c+c1}{\PYZsh{} or: mat @ mat}
\end{Verbatim}
\end{tcolorbox}

            \begin{tcolorbox}[breakable, size=fbox, boxrule=.5pt, pad at break*=1mm, opacityfill=0]
\prompt{Out}{outcolor}{32}{\boxspacing}
\begin{Verbatim}[commandchars=\\\{\}]
array([[ 15,  18,  21],
       [ 42,  54,  66],
       [ 69,  90, 111]])
\end{Verbatim}
\end{tcolorbox}
        
    \begin{tcolorbox}[breakable, size=fbox, boxrule=1pt, pad at break*=1mm,colback=cellbackground, colframe=cellborder]
\prompt{In}{incolor}{33}{\boxspacing}
\begin{Verbatim}[commandchars=\\\{\}]
\PY{c+c1}{\PYZsh{} vector dot product (returns a scalar)}
\PY{n}{np}\PY{o}{.}\PY{n}{dot}\PY{p}{(}\PY{n}{vec}\PY{p}{,} \PY{n}{vec}\PY{p}{)}    \PY{c+c1}{\PYZsh{} or: vec @ vec}
\end{Verbatim}
\end{tcolorbox}

            \begin{tcolorbox}[breakable, size=fbox, boxrule=.5pt, pad at break*=1mm, opacityfill=0]
\prompt{Out}{outcolor}{33}{\boxspacing}
\begin{Verbatim}[commandchars=\\\{\}]
5
\end{Verbatim}
\end{tcolorbox}
        
    \begin{tcolorbox}[breakable, size=fbox, boxrule=1pt, pad at break*=1mm,colback=cellbackground, colframe=cellborder]
\prompt{In}{incolor}{34}{\boxspacing}
\begin{Verbatim}[commandchars=\\\{\}]
\PY{c+c1}{\PYZsh{} matrix\PYZhy{}vector product (returns vector)}
\PY{n}{np}\PY{o}{.}\PY{n}{dot}\PY{p}{(}\PY{n}{mat}\PY{p}{,} \PY{n}{vec}\PY{p}{)}    \PY{c+c1}{\PYZsh{} or: mat @ vec}
\end{Verbatim}
\end{tcolorbox}

            \begin{tcolorbox}[breakable, size=fbox, boxrule=.5pt, pad at break*=1mm, opacityfill=0]
\prompt{Out}{outcolor}{34}{\boxspacing}
\begin{Verbatim}[commandchars=\\\{\}]
array([ 5, 14, 23])
\end{Verbatim}
\end{tcolorbox}
        
    We must of course make sure that matrices and vector have conformable
dimensions!

    \begin{tcolorbox}[breakable, size=fbox, boxrule=1pt, pad at break*=1mm,colback=cellbackground, colframe=cellborder]
\prompt{In}{incolor}{35}{\boxspacing}
\begin{Verbatim}[commandchars=\\\{\}]
\PY{n}{mat} \PY{o}{=} \PY{n}{np}\PY{o}{.}\PY{n}{arange}\PY{p}{(}\PY{l+m+mi}{6}\PY{p}{)}\PY{o}{.}\PY{n}{reshape}\PY{p}{(}\PY{p}{(}\PY{l+m+mi}{2}\PY{p}{,} \PY{l+m+mi}{3}\PY{p}{)}\PY{p}{)}
\PY{n}{mat}
\end{Verbatim}
\end{tcolorbox}

            \begin{tcolorbox}[breakable, size=fbox, boxrule=.5pt, pad at break*=1mm, opacityfill=0]
\prompt{Out}{outcolor}{35}{\boxspacing}
\begin{Verbatim}[commandchars=\\\{\}]
array([[0, 1, 2],
       [3, 4, 5]])
\end{Verbatim}
\end{tcolorbox}
        
    \begin{tcolorbox}[breakable, size=fbox, boxrule=1pt, pad at break*=1mm,colback=cellbackground, colframe=cellborder]
\prompt{In}{incolor}{36}{\boxspacing}
\begin{Verbatim}[commandchars=\\\{\}]
\PY{n}{np}\PY{o}{.}\PY{n}{dot}\PY{p}{(}\PY{n}{mat}\PY{p}{,} \PY{n}{mat}\PY{p}{)}        \PY{c+c1}{\PYZsh{} raises error, cannot multiply 2x3 matrix with 2x3 matrix}
\end{Verbatim}
\end{tcolorbox}

    \begin{Verbatim}[commandchars=\\\{\}, frame=single, framerule=2mm, rulecolor=\color{outerrorbackground}]
\textcolor{ansi-red}{---------------------------------------------------------------------------}
\textcolor{ansi-red}{ValueError}                                Traceback (most recent call last)
\textcolor{ansi-green}{<ipython-input-1-5a8b21790e76>} in \textcolor{ansi-cyan}{<module>}
\textcolor{ansi-green}{----> 1}\textcolor{ansi-red}{ }np\textcolor{ansi-blue}{.}dot\textcolor{ansi-blue}{(}mat\textcolor{ansi-blue}{,} mat\textcolor{ansi-blue}{)}        \textcolor{ansi-red}{\# raises error, cannot multiply 2x3 matrix with 2x3 matrix}

\textcolor{ansi-green}{<\_\_array\_function\_\_ internals>} in \textcolor{ansi-cyan}{dot}\textcolor{ansi-blue}{(*args, **kwargs)}

\textcolor{ansi-red}{ValueError}: shapes (2,3) and (2,3) not aligned: 3 (dim 1) != 2 (dim 0)
    \end{Verbatim}

    \begin{tcolorbox}[breakable, size=fbox, boxrule=1pt, pad at break*=1mm,colback=cellbackground, colframe=cellborder]
\prompt{In}{incolor}{37}{\boxspacing}
\begin{Verbatim}[commandchars=\\\{\}]
\PY{n}{np}\PY{o}{.}\PY{n}{dot}\PY{p}{(}\PY{n}{mat}\PY{p}{,} \PY{n}{mat}\PY{o}{.}\PY{n}{T}\PY{p}{)}      \PY{c+c1}{\PYZsh{} transpose second operand, works}
\end{Verbatim}
\end{tcolorbox}

            \begin{tcolorbox}[breakable, size=fbox, boxrule=.5pt, pad at break*=1mm, opacityfill=0]
\prompt{Out}{outcolor}{37}{\boxspacing}
\begin{Verbatim}[commandchars=\\\{\}]
array([[ 5, 14],
       [14, 50]])
\end{Verbatim}
\end{tcolorbox}
        
    \hypertarget{reductions}{%
\subsubsection{Reductions}\label{reductions}}

Reductions are operations that reduce the dimensionality of the data.
For example, computing the mean of an array reduces a collection of data
points to a single scalar, its mean.

Basic reduction operations include:

\begin{itemize}
\tightlist
\item
  \texttt{np.sum()}: sum of array elements
\item
  \texttt{np.prod()}: product of array elements
\item
  \texttt{np.amin()}, \texttt{np.amax()}: minimum and maximum element
\item
  \texttt{np.argmin()}, \texttt{np.argmax()}: location of minimum and
  maximum element
\item
  \texttt{np.mean()}, \texttt{np.average()}: mean of array elements
\item
  \texttt{np.median()}: median of array elements
\item
  \texttt{np.std()}, \texttt{np.var()}: standard deviation and variance
  of array elements
\item
  \texttt{np.percentile()}: percentiles of array elements
\end{itemize}

Most if not all reductions accept an \texttt{axis} argument which
restricts the operation to a specific axis.

\begin{itemize}
\tightlist
\item
  The resulting array will have one dimension less than the input.
\item
  If no axis is specified, the operation is applied to the whole
  (flattened) array.
\end{itemize}

\emph{Examples:}

    \begin{tcolorbox}[breakable, size=fbox, boxrule=1pt, pad at break*=1mm,colback=cellbackground, colframe=cellborder]
\prompt{In}{incolor}{38}{\boxspacing}
\begin{Verbatim}[commandchars=\\\{\}]
\PY{k+kn}{import} \PY{n+nn}{numpy} \PY{k}{as} \PY{n+nn}{np}

\PY{c+c1}{\PYZsh{} 1\PYZhy{}dimensional input data}
\PY{n}{data} \PY{o}{=} \PY{n}{np}\PY{o}{.}\PY{n}{linspace}\PY{p}{(}\PY{l+m+mf}{0.0}\PY{p}{,} \PY{l+m+mf}{1.0}\PY{p}{,} \PY{l+m+mi}{11}\PY{p}{)}

\PY{c+c1}{\PYZsh{} Compute mean and std. of input data}
\PY{n}{m} \PY{o}{=} \PY{n}{np}\PY{o}{.}\PY{n}{mean}\PY{p}{(}\PY{n}{data}\PY{p}{)}
\PY{n}{s} \PY{o}{=} \PY{n}{np}\PY{o}{.}\PY{n}{std}\PY{p}{(}\PY{n}{data}\PY{p}{)}
\PY{n+nb}{print}\PY{p}{(}\PY{l+s+s1}{\PYZsq{}}\PY{l+s+s1}{Mean: }\PY{l+s+si}{\PYZob{}:.2f\PYZcb{}}\PY{l+s+s1}{, std. dev.: }\PY{l+s+si}{\PYZob{}:.2f\PYZcb{}}\PY{l+s+s1}{\PYZsq{}}\PY{o}{.}\PY{n}{format}\PY{p}{(}\PY{n}{m}\PY{p}{,} \PY{n}{s}\PY{p}{)}\PY{p}{)}
\end{Verbatim}
\end{tcolorbox}

    \begin{Verbatim}[commandchars=\\\{\}]
Mean: 0.50, std. dev.: 0.32
    \end{Verbatim}

    \begin{tcolorbox}[breakable, size=fbox, boxrule=1pt, pad at break*=1mm,colback=cellbackground, colframe=cellborder]
\prompt{In}{incolor}{39}{\boxspacing}
\begin{Verbatim}[commandchars=\\\{\}]
\PY{c+c1}{\PYZsh{} 2\PYZhy{}dimensional input data}
\PY{n}{data} \PY{o}{=} \PY{n}{np}\PY{o}{.}\PY{n}{linspace}\PY{p}{(}\PY{l+m+mf}{0.0}\PY{p}{,} \PY{l+m+mf}{1.0}\PY{p}{,} \PY{l+m+mi}{21}\PY{p}{)}\PY{o}{.}\PY{n}{reshape}\PY{p}{(}\PY{p}{(}\PY{l+m+mi}{3}\PY{p}{,} \PY{l+m+mi}{7}\PY{p}{)}\PY{p}{)}
\PY{n}{data}
\end{Verbatim}
\end{tcolorbox}

            \begin{tcolorbox}[breakable, size=fbox, boxrule=.5pt, pad at break*=1mm, opacityfill=0]
\prompt{Out}{outcolor}{39}{\boxspacing}
\begin{Verbatim}[commandchars=\\\{\}]
array([[0.  , 0.05, 0.1 , 0.15, 0.2 , 0.25, 0.3 ],
       [0.35, 0.4 , 0.45, 0.5 , 0.55, 0.6 , 0.65],
       [0.7 , 0.75, 0.8 , 0.85, 0.9 , 0.95, 1.  ]])
\end{Verbatim}
\end{tcolorbox}
        
    \begin{tcolorbox}[breakable, size=fbox, boxrule=1pt, pad at break*=1mm,colback=cellbackground, colframe=cellborder]
\prompt{In}{incolor}{40}{\boxspacing}
\begin{Verbatim}[commandchars=\\\{\}]
\PY{c+c1}{\PYZsh{} Compute mean of each row, ie along the column axis}
\PY{n}{m} \PY{o}{=} \PY{n}{np}\PY{o}{.}\PY{n}{mean}\PY{p}{(}\PY{n}{data}\PY{p}{,} \PY{n}{axis}\PY{o}{=}\PY{l+m+mi}{1}\PY{p}{)}
\PY{n}{m}           \PY{c+c1}{\PYZsh{} Result is a vector of 3 elements, one for each row}
\end{Verbatim}
\end{tcolorbox}

            \begin{tcolorbox}[breakable, size=fbox, boxrule=.5pt, pad at break*=1mm, opacityfill=0]
\prompt{Out}{outcolor}{40}{\boxspacing}
\begin{Verbatim}[commandchars=\\\{\}]
array([0.15, 0.5 , 0.85])
\end{Verbatim}
\end{tcolorbox}
        
    \hypertarget{broadcasting-advanced}{%
\subsubsection{Broadcasting
{[}advanced{]}}\label{broadcasting-advanced}}

Element-wise operations in most programming languages require input
arrays to have identical shapes. NumPy relaxes this constraint and
allows us to use arrays with different shapes that can be ``broadcast''
to identical shapes.

What do we mean by ``broadcasting''?

\begin{itemize}
\tightlist
\item
  Imagine we want to add a \(2 \times 3\) matrix to a length-2 vector.
\item
  This operation does not make sense, unless we interpret the (column)
  vector as a \(2 \times 1\) matrix, and replicate it 3 times to obtain
  a \(2 \times 3\) matrix. This is exactly what NumPy does.
\end{itemize}

\emph{Examples:}

    \begin{tcolorbox}[breakable, size=fbox, boxrule=1pt, pad at break*=1mm,colback=cellbackground, colframe=cellborder]
\prompt{In}{incolor}{41}{\boxspacing}
\begin{Verbatim}[commandchars=\\\{\}]
\PY{k+kn}{import} \PY{n+nn}{numpy} \PY{k}{as} \PY{n+nn}{np}

\PY{c+c1}{\PYZsh{} Create 3x2 matrix}
\PY{n}{mat} \PY{o}{=} \PY{n}{np}\PY{o}{.}\PY{n}{arange}\PY{p}{(}\PY{l+m+mi}{6}\PY{p}{)}\PY{o}{.}\PY{n}{reshape}\PY{p}{(}\PY{p}{(}\PY{l+m+mi}{2}\PY{p}{,} \PY{l+m+mi}{3}\PY{p}{)}\PY{p}{)}
\PY{n}{mat}
\end{Verbatim}
\end{tcolorbox}

            \begin{tcolorbox}[breakable, size=fbox, boxrule=.5pt, pad at break*=1mm, opacityfill=0]
\prompt{Out}{outcolor}{41}{\boxspacing}
\begin{Verbatim}[commandchars=\\\{\}]
array([[0, 1, 2],
       [3, 4, 5]])
\end{Verbatim}
\end{tcolorbox}
        
    \begin{tcolorbox}[breakable, size=fbox, boxrule=1pt, pad at break*=1mm,colback=cellbackground, colframe=cellborder]
\prompt{In}{incolor}{42}{\boxspacing}
\begin{Verbatim}[commandchars=\\\{\}]
\PY{c+c1}{\PYZsh{} Create 2\PYZhy{}element vector}
\PY{n}{vec} \PY{o}{=} \PY{n}{np}\PY{o}{.}\PY{n}{arange}\PY{p}{(}\PY{l+m+mi}{2}\PY{p}{)}
\PY{n}{vec}
\end{Verbatim}
\end{tcolorbox}

            \begin{tcolorbox}[breakable, size=fbox, boxrule=.5pt, pad at break*=1mm, opacityfill=0]
\prompt{Out}{outcolor}{42}{\boxspacing}
\begin{Verbatim}[commandchars=\\\{\}]
array([0, 1])
\end{Verbatim}
\end{tcolorbox}
        
    \begin{tcolorbox}[breakable, size=fbox, boxrule=1pt, pad at break*=1mm,colback=cellbackground, colframe=cellborder]
\prompt{In}{incolor}{43}{\boxspacing}
\begin{Verbatim}[commandchars=\\\{\}]
\PY{c+c1}{\PYZsh{} Trying to add matrix to vector fails}
\PY{n}{mat} \PY{o}{+} \PY{n}{vec}
\end{Verbatim}
\end{tcolorbox}

    \begin{Verbatim}[commandchars=\\\{\}, frame=single, framerule=2mm, rulecolor=\color{outerrorbackground}]
\textcolor{ansi-red}{---------------------------------------------------------------------------}
\textcolor{ansi-red}{ValueError}                                Traceback (most recent call last)
\textcolor{ansi-green}{<ipython-input-1-d51c7a88afa4>} in \textcolor{ansi-cyan}{<module>}
\textcolor{ansi-green-intense}{\textbf{      1}} \textcolor{ansi-red}{\# Trying to add matrix to vector fails}
\textcolor{ansi-green}{----> 2}\textcolor{ansi-red}{ }mat \textcolor{ansi-blue}{+} vec

\textcolor{ansi-red}{ValueError}: operands could not be broadcast together with shapes (2,3) (2,) 
    \end{Verbatim}

    \begin{tcolorbox}[breakable, size=fbox, boxrule=1pt, pad at break*=1mm,colback=cellbackground, colframe=cellborder]
\prompt{In}{incolor}{44}{\boxspacing}
\begin{Verbatim}[commandchars=\\\{\}]
\PY{c+c1}{\PYZsh{} However, we can explicitly reshape the vector to a 2x1 column vector}
\PY{n}{colvec} \PY{o}{=} \PY{n}{vec}\PY{o}{.}\PY{n}{reshape}\PY{p}{(}\PY{p}{(}\PY{o}{\PYZhy{}}\PY{l+m+mi}{1}\PY{p}{,} \PY{l+m+mi}{1}\PY{p}{)}\PY{p}{)}
\PY{n}{colvec}
\end{Verbatim}
\end{tcolorbox}

            \begin{tcolorbox}[breakable, size=fbox, boxrule=.5pt, pad at break*=1mm, opacityfill=0]
\prompt{Out}{outcolor}{44}{\boxspacing}
\begin{Verbatim}[commandchars=\\\{\}]
array([[0],
       [1]])
\end{Verbatim}
\end{tcolorbox}
        
    \begin{tcolorbox}[breakable, size=fbox, boxrule=1pt, pad at break*=1mm,colback=cellbackground, colframe=cellborder]
\prompt{In}{incolor}{45}{\boxspacing}
\begin{Verbatim}[commandchars=\\\{\}]
\PY{c+c1}{\PYZsh{} Now, broadcasting replicates column vector to match matrix columns}
\PY{n}{mat} \PY{o}{+} \PY{n}{colvec}
\end{Verbatim}
\end{tcolorbox}

            \begin{tcolorbox}[breakable, size=fbox, boxrule=.5pt, pad at break*=1mm, opacityfill=0]
\prompt{Out}{outcolor}{45}{\boxspacing}
\begin{Verbatim}[commandchars=\\\{\}]
array([[0, 1, 2],
       [4, 5, 6]])
\end{Verbatim}
\end{tcolorbox}
        
    We do not need to \texttt{reshape()} data, but can instead use a feature
of NumPy that allows us to increase the number of dimensions on the
spot:

    \begin{tcolorbox}[breakable, size=fbox, boxrule=1pt, pad at break*=1mm,colback=cellbackground, colframe=cellborder]
\prompt{In}{incolor}{46}{\boxspacing}
\begin{Verbatim}[commandchars=\\\{\}]
\PY{c+c1}{\PYZsh{} use vec[:, None] to append an additional dimension to vec}
\PY{n}{mat} \PY{o}{+} \PY{n}{vec}\PY{p}{[}\PY{p}{:}\PY{p}{,} \PY{k+kc}{None}\PY{p}{]}
\end{Verbatim}
\end{tcolorbox}

            \begin{tcolorbox}[breakable, size=fbox, boxrule=.5pt, pad at break*=1mm, opacityfill=0]
\prompt{Out}{outcolor}{46}{\boxspacing}
\begin{Verbatim}[commandchars=\\\{\}]
array([[0, 1, 2],
       [4, 5, 6]])
\end{Verbatim}
\end{tcolorbox}
        
    For more examples, see the official NumPy
\href{https://numpy.org/doc/stable/user/theory.broadcasting.html}{tutorial}
on broadcasting.

    \begin{center}\rule{0.5\linewidth}{0.5pt}\end{center}

\hypertarget{numpy-data-types-advanced}{%
\subsection{NumPy data types
{[}advanced{]}}\label{numpy-data-types-advanced}}

\hypertarget{default-data-types}{%
\subsubsection{Default data types}\label{default-data-types}}

We have already touched upon the numerical data types used by NumPy. If
we do not explicitly request a data type using the \texttt{dtype}
keyword argument, NumPy by default behaves as follows:

\begin{enumerate}
\def\labelenumi{\arabic{enumi}.}
\item
  The default data type for most array creation routines which create
  arrays of a given shape or size, such as \texttt{np.empty()},
  \texttt{np.ones()} and \texttt{np.zeros()}, is a 64-bit floating-point
  number (\texttt{np.float64}).
\item
  Array creation routines that accept numerical input data will use the
  data type of this input data to determine the array data type.

  Examples of such functions are \texttt{np.arange()} and
  \texttt{np.array()}.
\item
  Arrays that are implicitly created as a result of an operation
  (addition, etc.) are assigned the most suitable type to represent the
  result.

  For example, when adding a floating-point and an integer array, the
  result will be a floating-point array.
\end{enumerate}

\emph{Examples:}

\emph{Case 1:} default data type is \texttt{np.float64}:

    \begin{tcolorbox}[breakable, size=fbox, boxrule=1pt, pad at break*=1mm,colback=cellbackground, colframe=cellborder]
\prompt{In}{incolor}{47}{\boxspacing}
\begin{Verbatim}[commandchars=\\\{\}]
\PY{k+kn}{import} \PY{n+nn}{numpy} \PY{k}{as} \PY{n+nn}{np}

\PY{n}{x} \PY{o}{=} \PY{n}{np}\PY{o}{.}\PY{n}{ones}\PY{p}{(}\PY{l+m+mi}{1}\PY{p}{)}      \PY{c+c1}{\PYZsh{} length\PYZhy{}1 vector of ones}
\PY{n}{x}\PY{o}{.}\PY{n}{dtype}             \PY{c+c1}{\PYZsh{} default type: float64}
\end{Verbatim}
\end{tcolorbox}

            \begin{tcolorbox}[breakable, size=fbox, boxrule=.5pt, pad at break*=1mm, opacityfill=0]
\prompt{Out}{outcolor}{47}{\boxspacing}
\begin{Verbatim}[commandchars=\\\{\}]
dtype('float64')
\end{Verbatim}
\end{tcolorbox}
        
    \emph{Case 2:} data type depends on input data:

    \begin{tcolorbox}[breakable, size=fbox, boxrule=1pt, pad at break*=1mm,colback=cellbackground, colframe=cellborder]
\prompt{In}{incolor}{48}{\boxspacing}
\begin{Verbatim}[commandchars=\\\{\}]
\PY{c+c1}{\PYZsh{} Argument is an integer}
\PY{n}{x} \PY{o}{=} \PY{n}{np}\PY{o}{.}\PY{n}{arange}\PY{p}{(}\PY{l+m+mi}{5}\PY{p}{)}
\PY{n}{x}\PY{o}{.}\PY{n}{dtype}             \PY{c+c1}{\PYZsh{} data type is np.int64 (on most platforms)}
\end{Verbatim}
\end{tcolorbox}

            \begin{tcolorbox}[breakable, size=fbox, boxrule=.5pt, pad at break*=1mm, opacityfill=0]
\prompt{Out}{outcolor}{48}{\boxspacing}
\begin{Verbatim}[commandchars=\\\{\}]
dtype('int64')
\end{Verbatim}
\end{tcolorbox}
        
    \begin{tcolorbox}[breakable, size=fbox, boxrule=1pt, pad at break*=1mm,colback=cellbackground, colframe=cellborder]
\prompt{In}{incolor}{49}{\boxspacing}
\begin{Verbatim}[commandchars=\\\{\}]
\PY{c+c1}{\PYZsh{} Argument is a float}
\PY{n}{x} \PY{o}{=} \PY{n}{np}\PY{o}{.}\PY{n}{arange}\PY{p}{(}\PY{l+m+mf}{5.0}\PY{p}{)}
\PY{n}{x}\PY{o}{.}\PY{n}{dtype}             \PY{c+c1}{\PYZsh{} data type is np.float64}
\end{Verbatim}
\end{tcolorbox}

            \begin{tcolorbox}[breakable, size=fbox, boxrule=.5pt, pad at break*=1mm, opacityfill=0]
\prompt{Out}{outcolor}{49}{\boxspacing}
\begin{Verbatim}[commandchars=\\\{\}]
dtype('float64')
\end{Verbatim}
\end{tcolorbox}
        
    \emph{Case 3:} data type determined to accommodate result

    \begin{tcolorbox}[breakable, size=fbox, boxrule=1pt, pad at break*=1mm,colback=cellbackground, colframe=cellborder]
\prompt{In}{incolor}{50}{\boxspacing}
\begin{Verbatim}[commandchars=\\\{\}]
\PY{c+c1}{\PYZsh{} Add two integer arrays}
\PY{n}{arr1} \PY{o}{=} \PY{n}{np}\PY{o}{.}\PY{n}{arange}\PY{p}{(}\PY{l+m+mi}{3}\PY{p}{)}
\PY{n}{arr2} \PY{o}{=} \PY{n}{np}\PY{o}{.}\PY{n}{arange}\PY{p}{(}\PY{l+m+mi}{3}\PY{p}{,} \PY{l+m+mi}{0}\PY{p}{,} \PY{o}{\PYZhy{}}\PY{l+m+mi}{1}\PY{p}{)}      \PY{c+c1}{\PYZsh{} creates [3, 2, 1]}
\PY{n}{result} \PY{o}{=} \PY{n}{arr1} \PY{o}{+} \PY{n}{arr2}
\PY{n+nb}{print}\PY{p}{(}\PY{n}{result}\PY{p}{)}
\PY{n}{result}\PY{o}{.}\PY{n}{dtype}                    \PY{c+c1}{\PYZsh{} data type is np.int64}
\end{Verbatim}
\end{tcolorbox}

    \begin{Verbatim}[commandchars=\\\{\}]
[3 3 3]
    \end{Verbatim}

            \begin{tcolorbox}[breakable, size=fbox, boxrule=.5pt, pad at break*=1mm, opacityfill=0]
\prompt{Out}{outcolor}{50}{\boxspacing}
\begin{Verbatim}[commandchars=\\\{\}]
dtype('int64')
\end{Verbatim}
\end{tcolorbox}
        
    \begin{tcolorbox}[breakable, size=fbox, boxrule=1pt, pad at break*=1mm,colback=cellbackground, colframe=cellborder]
\prompt{In}{incolor}{51}{\boxspacing}
\begin{Verbatim}[commandchars=\\\{\}]
\PY{c+c1}{\PYZsh{} Add integer to floating\PYZhy{}point array}
\PY{n}{arr1} \PY{o}{=} \PY{n}{np}\PY{o}{.}\PY{n}{arange}\PY{p}{(}\PY{l+m+mi}{3}\PY{p}{)}
\PY{n}{arr2} \PY{o}{=} \PY{n}{np}\PY{o}{.}\PY{n}{arange}\PY{p}{(}\PY{l+m+mf}{3.0}\PY{p}{,} \PY{l+m+mf}{0.0}\PY{p}{,} \PY{o}{\PYZhy{}}\PY{l+m+mf}{1.0}\PY{p}{)}    \PY{c+c1}{\PYZsh{} creates [3.0, 2.0, 1.0]}
\PY{n}{result} \PY{o}{=} \PY{n}{arr1} \PY{o}{+} \PY{n}{arr2}
\PY{n+nb}{print}\PY{p}{(}\PY{n}{result}\PY{p}{)}
\PY{n}{result}\PY{o}{.}\PY{n}{dtype}                        \PY{c+c1}{\PYZsh{} data type is np.float64}
\end{Verbatim}
\end{tcolorbox}

    \begin{Verbatim}[commandchars=\\\{\}]
[3. 3. 3.]
    \end{Verbatim}

            \begin{tcolorbox}[breakable, size=fbox, boxrule=.5pt, pad at break*=1mm, opacityfill=0]
\prompt{Out}{outcolor}{51}{\boxspacing}
\begin{Verbatim}[commandchars=\\\{\}]
dtype('float64')
\end{Verbatim}
\end{tcolorbox}
        
    Even though the resulting array is \texttt{{[}3.0,\ 3.0,\ 3.0{]}} and
can thus be represented as integers without loss of data, NumPy only
takes into account that one of the operands is floating-point, and thus
the result has to be of floating-point type!

    \hypertarget{explicit-data-types}{%
\subsubsection{Explicit data types}\label{explicit-data-types}}

We can almost always explicitly request an array to be of a particular
data type by passing the \texttt{dtype} keyword argument. The most
common types are:

\begin{itemize}
\item
  \texttt{np.float64}: a 64-bit floating-point number, also called
  \emph{double precision} in other languages.

  This is the most commonly used floating-point data type. It can
  represent numbers with up to 16 decimal digits, and covers a range of
  approximately \(\pm 10^{308}\).
\item
  \texttt{np.int64}: a 64-bit integer which can represent integer values
  on the interval of (approximately) \(\pm10^{19}\).

  Unlike floating-point, the integer representation is \emph{exact}, but
  covers a much smaller range (and, obviously, no fractional numbers)
\item
  \texttt{np.float32}, \texttt{np.float16}: single-precision and
  half-precision floating-point numbers. These occupy only 32 and 16
  bits of memory, respectively.

  They thus trade off storage requirements for a loss of precision and
  range.
\item
  \texttt{np.int32}, \texttt{np.int16}, \texttt{np.int8} represent
  integers using 32, 16 and 8 bits, respectively.

  They require less memory, but can represent only a smaller range of
  integers. For example, \texttt{np.int8} can only store integer values
  from -128 to 127.
\item
  NumPy also supports complex numerical types to represent imaginary
  numbers. We will not be using those in this tutorial.
\end{itemize}

Would we ever want to use anything other than the default data types,
which in most cases are either \texttt{np.float64} and
\texttt{np.int64}? These, after all, support the largest range and
highest precision. This is true in general, but there are special cases
where other data types need to be used:

\begin{enumerate}
\def\labelenumi{\arabic{enumi}.}
\item
  \emph{Storage requirements:} if you work with large amounts of data,
  for example arrays with many dimensions, you can run out of memory or
  storage space (when saving results to files).

  In this case, you can store data as \texttt{np.float32} instead of
  \texttt{np.float64}, which halves the storage requirement.

  Similarly, if you know that your integer data only takes on values
  between -128 and 127, you can store them as \texttt{np.int8} which
  consumes only 1/8 of the space compared to \texttt{np.int64}!
\item
  \emph{Performance:} Some tasks simply don't require high precision or
  range. For example, some machine learning tasks can be performed using
  only 8-bit integers, and companies like Google have developed
  dedicated processors to considerably speed up workloads using 8-bit
  integers.

  Even if you are not using any specialised CPUs or GPUs, data has to be
  transferred from memory to the processor and this is a major
  performance bottleneck. The less data needs to be transferred, the
  better!

  In general, this is nothing you need to worry about at this point, but
  might become relevant once you start writing complex high-performance
  code.
\end{enumerate}

\emph{Examples:}

    \begin{tcolorbox}[breakable, size=fbox, boxrule=1pt, pad at break*=1mm,colback=cellbackground, colframe=cellborder]
\prompt{In}{incolor}{52}{\boxspacing}
\begin{Verbatim}[commandchars=\\\{\}]
\PY{k+kn}{import} \PY{n+nn}{numpy} \PY{k}{as} \PY{n+nn}{np}

\PY{c+c1}{\PYZsh{} Explicitly specify data type}
\PY{n}{x} \PY{o}{=} \PY{n}{np}\PY{o}{.}\PY{n}{ones}\PY{p}{(}\PY{l+m+mi}{1}\PY{p}{,} \PY{n}{dtype}\PY{o}{=}\PY{n}{np}\PY{o}{.}\PY{n}{float16}\PY{p}{)}
\PY{n}{x}       \PY{c+c1}{\PYZsh{} prints np.float16}
\end{Verbatim}
\end{tcolorbox}

            \begin{tcolorbox}[breakable, size=fbox, boxrule=.5pt, pad at break*=1mm, opacityfill=0]
\prompt{Out}{outcolor}{52}{\boxspacing}
\begin{Verbatim}[commandchars=\\\{\}]
array([1.], dtype=float16)
\end{Verbatim}
\end{tcolorbox}
        
    We can use \texttt{dtype} to override the data type inferred from input
data:

    \begin{tcolorbox}[breakable, size=fbox, boxrule=1pt, pad at break*=1mm,colback=cellbackground, colframe=cellborder]
\prompt{In}{incolor}{53}{\boxspacing}
\begin{Verbatim}[commandchars=\\\{\}]
\PY{n}{lst} \PY{o}{=} \PY{p}{[}\PY{l+m+mi}{1}\PY{p}{,} \PY{l+m+mi}{2}\PY{p}{,} \PY{l+m+mi}{3}\PY{p}{]}
\PY{n}{x} \PY{o}{=} \PY{n}{np}\PY{o}{.}\PY{n}{array}\PY{p}{(}\PY{n}{lst}\PY{p}{)}       \PY{c+c1}{\PYZsh{} given list of integers, creates integer array}
\PY{n}{x}\PY{o}{.}\PY{n}{dtype}                 \PY{c+c1}{\PYZsh{} prints np.int64}
\end{Verbatim}
\end{tcolorbox}

            \begin{tcolorbox}[breakable, size=fbox, boxrule=.5pt, pad at break*=1mm, opacityfill=0]
\prompt{Out}{outcolor}{53}{\boxspacing}
\begin{Verbatim}[commandchars=\\\{\}]
dtype('int64')
\end{Verbatim}
\end{tcolorbox}
        
    \begin{tcolorbox}[breakable, size=fbox, boxrule=1pt, pad at break*=1mm,colback=cellbackground, colframe=cellborder]
\prompt{In}{incolor}{54}{\boxspacing}
\begin{Verbatim}[commandchars=\\\{\}]
\PY{c+c1}{\PYZsh{} override inferred data type:}
\PY{c+c1}{\PYZsh{} created floating\PYZhy{}point array even if integers were given}
\PY{n}{x} \PY{o}{=} \PY{n}{np}\PY{o}{.}\PY{n}{array}\PY{p}{(}\PY{n}{lst}\PY{p}{,} \PY{n}{dtype}\PY{o}{=}\PY{n}{np}\PY{o}{.}\PY{n}{float64}\PY{p}{)}
\PY{n}{x}\PY{o}{.}\PY{n}{dtype}                 \PY{c+c1}{\PYZsh{} prints np.float64}
\end{Verbatim}
\end{tcolorbox}

            \begin{tcolorbox}[breakable, size=fbox, boxrule=.5pt, pad at break*=1mm, opacityfill=0]
\prompt{Out}{outcolor}{54}{\boxspacing}
\begin{Verbatim}[commandchars=\\\{\}]
dtype('float64')
\end{Verbatim}
\end{tcolorbox}
        
    \begin{tcolorbox}[breakable, size=fbox, boxrule=1pt, pad at break*=1mm,colback=cellbackground, colframe=cellborder]
\prompt{In}{incolor}{55}{\boxspacing}
\begin{Verbatim}[commandchars=\\\{\}]
\PY{c+c1}{\PYZsh{} override inferred data type:}
\PY{c+c1}{\PYZsh{} created integer array even if floats were given,}
\PY{c+c1}{\PYZsh{} thus truncating input data!}
\PY{n}{lst} \PY{o}{=} \PY{p}{[}\PY{l+m+mf}{1.234}\PY{p}{,} \PY{l+m+mf}{4.567}\PY{p}{,} \PY{l+m+mf}{6.789}\PY{p}{]}
\PY{n}{x} \PY{o}{=} \PY{n}{np}\PY{o}{.}\PY{n}{array}\PY{p}{(}\PY{n}{lst}\PY{p}{,} \PY{n}{dtype}\PY{o}{=}\PY{n}{np}\PY{o}{.}\PY{n}{int}\PY{p}{)}
\PY{n+nb}{print}\PY{p}{(}\PY{n}{x}\PY{p}{)}                \PY{c+c1}{\PYZsh{} prints [1, 4, 6]}
\PY{n}{x}\PY{o}{.}\PY{n}{dtype}                 \PY{c+c1}{\PYZsh{} prints np.int64}
\end{Verbatim}
\end{tcolorbox}

    \begin{Verbatim}[commandchars=\\\{\}]
[1 4 6]
    \end{Verbatim}

            \begin{tcolorbox}[breakable, size=fbox, boxrule=.5pt, pad at break*=1mm, opacityfill=0]
\prompt{Out}{outcolor}{55}{\boxspacing}
\begin{Verbatim}[commandchars=\\\{\}]
dtype('int64')
\end{Verbatim}
\end{tcolorbox}
        
    \begin{center}\rule{0.5\linewidth}{0.5pt}\end{center}

\hypertarget{copies-and-views-advanced}{%
\subsection{Copies and views
{[}advanced{]}}\label{copies-and-views-advanced}}

Recall that assignment in Python does \emph{not} create a copy (unlike
in C, Fortran or Matlab):

    \begin{tcolorbox}[breakable, size=fbox, boxrule=1pt, pad at break*=1mm,colback=cellbackground, colframe=cellborder]
\prompt{In}{incolor}{56}{\boxspacing}
\begin{Verbatim}[commandchars=\\\{\}]
\PY{n}{a} \PY{o}{=} \PY{p}{[}\PY{l+m+mi}{0}\PY{p}{,} \PY{l+m+mi}{0}\PY{p}{,} \PY{l+m+mi}{0}\PY{p}{]}
\PY{n}{b} \PY{o}{=} \PY{n}{a}           \PY{c+c1}{\PYZsh{} b references the same object as a}
\PY{n}{b}\PY{p}{[}\PY{l+m+mi}{1}\PY{p}{]} \PY{o}{=} \PY{l+m+mi}{1}        \PY{c+c1}{\PYZsh{} modify second element of b (and a!)}
\PY{n}{a} \PY{o}{==} \PY{n}{b}          \PY{c+c1}{\PYZsh{} a and b are still the same}
\end{Verbatim}
\end{tcolorbox}

            \begin{tcolorbox}[breakable, size=fbox, boxrule=.5pt, pad at break*=1mm, opacityfill=0]
\prompt{Out}{outcolor}{56}{\boxspacing}
\begin{Verbatim}[commandchars=\\\{\}]
True
\end{Verbatim}
\end{tcolorbox}
        
    NumPy adds another layer to this sort of data sharing: whenever you
perform an assignment or indexing operation, NumPy tries hard \emph{not}
to copy the underlying data but instead creates a so-called view which
points to the same block of memory. It does so for performance reasons
(copying is expensive).

We can illustrate this using array slicing:

    \begin{tcolorbox}[breakable, size=fbox, boxrule=1pt, pad at break*=1mm,colback=cellbackground, colframe=cellborder]
\prompt{In}{incolor}{57}{\boxspacing}
\begin{Verbatim}[commandchars=\\\{\}]
\PY{k+kn}{import} \PY{n+nn}{numpy} \PY{k}{as} \PY{n+nn}{np}

\PY{n}{x} \PY{o}{=} \PY{n}{np}\PY{o}{.}\PY{n}{arange}\PY{p}{(}\PY{l+m+mi}{10}\PY{p}{)}
\PY{n}{x}
\end{Verbatim}
\end{tcolorbox}

            \begin{tcolorbox}[breakable, size=fbox, boxrule=.5pt, pad at break*=1mm, opacityfill=0]
\prompt{Out}{outcolor}{57}{\boxspacing}
\begin{Verbatim}[commandchars=\\\{\}]
array([0, 1, 2, 3, 4, 5, 6, 7, 8, 9])
\end{Verbatim}
\end{tcolorbox}
        
    \begin{tcolorbox}[breakable, size=fbox, boxrule=1pt, pad at break*=1mm,colback=cellbackground, colframe=cellborder]
\prompt{In}{incolor}{58}{\boxspacing}
\begin{Verbatim}[commandchars=\\\{\}]
\PY{n}{y} \PY{o}{=} \PY{n}{x}\PY{p}{[}\PY{l+m+mi}{3}\PY{p}{:}\PY{l+m+mi}{8}\PY{p}{]}          \PY{c+c1}{\PYZsh{} Create array that points to elements 4\PYZhy{}8 of x}
\PY{n}{y}
\end{Verbatim}
\end{tcolorbox}

            \begin{tcolorbox}[breakable, size=fbox, boxrule=.5pt, pad at break*=1mm, opacityfill=0]
\prompt{Out}{outcolor}{58}{\boxspacing}
\begin{Verbatim}[commandchars=\\\{\}]
array([3, 4, 5, 6, 7])
\end{Verbatim}
\end{tcolorbox}
        
    \texttt{x} and \texttt{y} are two different Python objects, which we can
verify using the built-in \texttt{id()} function:

    \begin{tcolorbox}[breakable, size=fbox, boxrule=1pt, pad at break*=1mm,colback=cellbackground, colframe=cellborder]
\prompt{In}{incolor}{59}{\boxspacing}
\begin{Verbatim}[commandchars=\\\{\}]
\PY{n+nb}{print}\PY{p}{(}\PY{n+nb}{id}\PY{p}{(}\PY{n}{x}\PY{p}{)}\PY{p}{)}
\PY{n+nb}{print}\PY{p}{(}\PY{n+nb}{id}\PY{p}{(}\PY{n}{y}\PY{p}{)}\PY{p}{)}
\end{Verbatim}
\end{tcolorbox}

    \begin{Verbatim}[commandchars=\\\{\}]
140335487563856
140335181380224
    \end{Verbatim}

    \begin{tcolorbox}[breakable, size=fbox, boxrule=1pt, pad at break*=1mm,colback=cellbackground, colframe=cellborder]
\prompt{In}{incolor}{60}{\boxspacing}
\begin{Verbatim}[commandchars=\\\{\}]
\PY{n+nb}{id}\PY{p}{(}\PY{n}{x}\PY{p}{)} \PY{o}{==} \PY{n+nb}{id}\PY{p}{(}\PY{n}{y}\PY{p}{)}
\end{Verbatim}
\end{tcolorbox}

            \begin{tcolorbox}[breakable, size=fbox, boxrule=.5pt, pad at break*=1mm, opacityfill=0]
\prompt{Out}{outcolor}{60}{\boxspacing}
\begin{Verbatim}[commandchars=\\\{\}]
False
\end{Verbatim}
\end{tcolorbox}
        
    And yet, the NumPy implementation makes sure that they reference the
same block of memory!

We can see this easily by modifying \texttt{y}:

    \begin{tcolorbox}[breakable, size=fbox, boxrule=1pt, pad at break*=1mm,colback=cellbackground, colframe=cellborder]
\prompt{In}{incolor}{61}{\boxspacing}
\begin{Verbatim}[commandchars=\\\{\}]
\PY{n}{y}\PY{p}{[}\PY{p}{:}\PY{p}{]} \PY{o}{=} \PY{l+m+mi}{0}        \PY{c+c1}{\PYZsh{} overwrite all elements of y with zeros}
\PY{n}{y}
\end{Verbatim}
\end{tcolorbox}

            \begin{tcolorbox}[breakable, size=fbox, boxrule=.5pt, pad at break*=1mm, opacityfill=0]
\prompt{Out}{outcolor}{61}{\boxspacing}
\begin{Verbatim}[commandchars=\\\{\}]
array([0, 0, 0, 0, 0])
\end{Verbatim}
\end{tcolorbox}
        
    \begin{tcolorbox}[breakable, size=fbox, boxrule=1pt, pad at break*=1mm,colback=cellbackground, colframe=cellborder]
\prompt{In}{incolor}{62}{\boxspacing}
\begin{Verbatim}[commandchars=\\\{\}]
\PY{n}{x}               \PY{c+c1}{\PYZsh{} elements of x that are also referenced by y}
                \PY{c+c1}{\PYZsh{} are now also zero!}
\end{Verbatim}
\end{tcolorbox}

            \begin{tcolorbox}[breakable, size=fbox, boxrule=.5pt, pad at break*=1mm, opacityfill=0]
\prompt{Out}{outcolor}{62}{\boxspacing}
\begin{Verbatim}[commandchars=\\\{\}]
array([0, 1, 2, 0, 0, 0, 0, 0, 8, 9])
\end{Verbatim}
\end{tcolorbox}
        
    This behaviour is even triggered when \texttt{y} references non-adjacent
elements in \texttt{x}. For example, we can let \texttt{y} be a view
onto every \emph{second} element in \texttt{x}:

    \begin{tcolorbox}[breakable, size=fbox, boxrule=1pt, pad at break*=1mm,colback=cellbackground, colframe=cellborder]
\prompt{In}{incolor}{63}{\boxspacing}
\begin{Verbatim}[commandchars=\\\{\}]
\PY{n}{x} \PY{o}{=} \PY{n}{np}\PY{o}{.}\PY{n}{arange}\PY{p}{(}\PY{l+m+mi}{10}\PY{p}{)}
\PY{n}{y} \PY{o}{=} \PY{n}{x}\PY{p}{[}\PY{p}{:}\PY{p}{:}\PY{l+m+mi}{2}\PY{p}{]}      \PY{c+c1}{\PYZsh{} y now points to every second element of x}
\PY{n}{y}\PY{p}{[}\PY{p}{:}\PY{p}{]} \PY{o}{=} \PY{l+m+mi}{0}        \PY{c+c1}{\PYZsh{} overwrite all elements of y with zeros}
\PY{n}{x}               \PY{c+c1}{\PYZsh{} every second element in x is now zero!}
\end{Verbatim}
\end{tcolorbox}

            \begin{tcolorbox}[breakable, size=fbox, boxrule=.5pt, pad at break*=1mm, opacityfill=0]
\prompt{Out}{outcolor}{63}{\boxspacing}
\begin{Verbatim}[commandchars=\\\{\}]
array([0, 1, 0, 3, 0, 5, 0, 7, 0, 9])
\end{Verbatim}
\end{tcolorbox}
        
    As a rule of thumb, NumPy will create a view as opposed to copying data
if

\begin{itemize}
\tightlist
\item
  An array is created from another array via slicing (ie. indexing using
  the \texttt{start:stop:step} triplet)
\end{itemize}

Conversely, a \emph{copy} is created whenever

\begin{itemize}
\tightlist
\item
  An array is created from another array via boolean (mask) indexing.
\item
  An array is created from another array via integer array indexing.
\end{itemize}

You can always force NumPy to create a copy by calling
\texttt{np.copy()}!

\emph{Examples:}

    \begin{tcolorbox}[breakable, size=fbox, boxrule=1pt, pad at break*=1mm,colback=cellbackground, colframe=cellborder]
\prompt{In}{incolor}{64}{\boxspacing}
\begin{Verbatim}[commandchars=\\\{\}]
\PY{c+c1}{\PYZsh{} Copies are created with boolean indexing}
\PY{n}{x} \PY{o}{=} \PY{n}{np}\PY{o}{.}\PY{n}{arange}\PY{p}{(}\PY{l+m+mi}{10}\PY{p}{)}
\PY{n}{mask} \PY{o}{=} \PY{p}{(}\PY{n}{x} \PY{o}{\PYZgt{}} \PY{l+m+mi}{4}\PY{p}{)}      \PY{c+c1}{\PYZsh{} boolean mask}
\PY{n}{y} \PY{o}{=} \PY{n}{x}\PY{p}{[}\PY{n}{mask}\PY{p}{]}         \PY{c+c1}{\PYZsh{} create y using boolean indexing}
\PY{n}{y}\PY{p}{[}\PY{p}{:}\PY{p}{]} \PY{o}{=} \PY{l+m+mi}{0}
\PY{n}{x}                   \PY{c+c1}{\PYZsh{} x is unmodified}
\end{Verbatim}
\end{tcolorbox}

            \begin{tcolorbox}[breakable, size=fbox, boxrule=.5pt, pad at break*=1mm, opacityfill=0]
\prompt{Out}{outcolor}{64}{\boxspacing}
\begin{Verbatim}[commandchars=\\\{\}]
array([0, 1, 2, 3, 4, 5, 6, 7, 8, 9])
\end{Verbatim}
\end{tcolorbox}
        
    \begin{tcolorbox}[breakable, size=fbox, boxrule=1pt, pad at break*=1mm,colback=cellbackground, colframe=cellborder]
\prompt{In}{incolor}{65}{\boxspacing}
\begin{Verbatim}[commandchars=\\\{\}]
\PY{c+c1}{\PYZsh{} Copies are created with integer array indexing}
\PY{n}{x} \PY{o}{=} \PY{n}{np}\PY{o}{.}\PY{n}{arange}\PY{p}{(}\PY{l+m+mi}{10}\PY{p}{)}
\PY{n}{index} \PY{o}{=} \PY{p}{[}\PY{l+m+mi}{3}\PY{p}{,} \PY{l+m+mi}{4}\PY{p}{,} \PY{l+m+mi}{5}\PY{p}{]}   \PY{c+c1}{\PYZsh{} List of indices to include in y}
\PY{n}{y} \PY{o}{=} \PY{n}{x}\PY{p}{[}\PY{n}{index}\PY{p}{]}
\PY{n}{y}\PY{p}{[}\PY{p}{:}\PY{p}{]} \PY{o}{=} \PY{l+m+mi}{0}
\PY{n}{x}                   \PY{c+c1}{\PYZsh{} x is unmodified}
\end{Verbatim}
\end{tcolorbox}

            \begin{tcolorbox}[breakable, size=fbox, boxrule=.5pt, pad at break*=1mm, opacityfill=0]
\prompt{Out}{outcolor}{65}{\boxspacing}
\begin{Verbatim}[commandchars=\\\{\}]
array([0, 1, 2, 3, 4, 5, 6, 7, 8, 9])
\end{Verbatim}
\end{tcolorbox}
        
    \begin{tcolorbox}[breakable, size=fbox, boxrule=1pt, pad at break*=1mm,colback=cellbackground, colframe=cellborder]
\prompt{In}{incolor}{66}{\boxspacing}
\begin{Verbatim}[commandchars=\\\{\}]
\PY{c+c1}{\PYZsh{} Forced copy with slicing}
\PY{n}{x} \PY{o}{=} \PY{n}{np}\PY{o}{.}\PY{n}{arange}\PY{p}{(}\PY{l+m+mi}{10}\PY{p}{)}
\PY{n}{y} \PY{o}{=} \PY{n}{np}\PY{o}{.}\PY{n}{copy}\PY{p}{(}\PY{n}{x}\PY{p}{[}\PY{l+m+mi}{3}\PY{p}{:}\PY{l+m+mi}{8}\PY{p}{]}\PY{p}{)} \PY{c+c1}{\PYZsh{} force copy with np.copy()}
\PY{n}{y}\PY{p}{[}\PY{p}{:}\PY{p}{]} \PY{o}{=} \PY{l+m+mi}{0}
\PY{n}{x}                   \PY{c+c1}{\PYZsh{} x is unmodified}
\end{Verbatim}
\end{tcolorbox}

            \begin{tcolorbox}[breakable, size=fbox, boxrule=.5pt, pad at break*=1mm, opacityfill=0]
\prompt{Out}{outcolor}{66}{\boxspacing}
\begin{Verbatim}[commandchars=\\\{\}]
array([0, 1, 2, 3, 4, 5, 6, 7, 8, 9])
\end{Verbatim}
\end{tcolorbox}
        
    As an alternative to \texttt{np.copy()} we can directly call the
\texttt{copy()} method of an array:

    \begin{tcolorbox}[breakable, size=fbox, boxrule=1pt, pad at break*=1mm,colback=cellbackground, colframe=cellborder]
\prompt{In}{incolor}{67}{\boxspacing}
\begin{Verbatim}[commandchars=\\\{\}]
\PY{n}{y} \PY{o}{=} \PY{n}{x}\PY{p}{[}\PY{l+m+mi}{3}\PY{p}{:}\PY{l+m+mi}{8}\PY{p}{]}\PY{o}{.}\PY{n}{copy}\PY{p}{(}\PY{p}{)}
\end{Verbatim}
\end{tcolorbox}

    \begin{center}\rule{0.5\linewidth}{0.5pt}\end{center}

\hypertarget{array-storage-order-advanced}{%
\subsection{Array storage order
{[}advanced{]}}\label{array-storage-order-advanced}}

Computer memory is linear, so a multi-dimensional array is mapped to a
one-dimensional block in memory. This can be done in two ways:

\begin{enumerate}
\def\labelenumi{\arabic{enumi}.}
\tightlist
\item
  NumPy uses the so-called \emph{row-major order} (also called \emph{C
  order}, because its the same as in C programming language)
\item
  This is exactly the opposite of Matlab, which uses \emph{column-major
  order} (also called \emph{F order}, because its the same as in the
  Fortran programming language)
\end{enumerate}

    \begin{tcolorbox}[breakable, size=fbox, boxrule=1pt, pad at break*=1mm,colback=cellbackground, colframe=cellborder]
\prompt{In}{incolor}{68}{\boxspacing}
\begin{Verbatim}[commandchars=\\\{\}]
\PY{k+kn}{import} \PY{n+nn}{numpy} \PY{k}{as} \PY{n+nn}{np}

\PY{n}{mat} \PY{o}{=} \PY{n}{np}\PY{o}{.}\PY{n}{arange}\PY{p}{(}\PY{l+m+mi}{6}\PY{p}{)}\PY{o}{.}\PY{n}{reshape}\PY{p}{(}\PY{p}{(}\PY{l+m+mi}{2}\PY{p}{,}\PY{l+m+mi}{3}\PY{p}{)}\PY{p}{)}
\PY{n}{mat}
\end{Verbatim}
\end{tcolorbox}

            \begin{tcolorbox}[breakable, size=fbox, boxrule=.5pt, pad at break*=1mm, opacityfill=0]
\prompt{Out}{outcolor}{68}{\boxspacing}
\begin{Verbatim}[commandchars=\\\{\}]
array([[0, 1, 2],
       [3, 4, 5]])
\end{Verbatim}
\end{tcolorbox}
        
    \begin{tcolorbox}[breakable, size=fbox, boxrule=1pt, pad at break*=1mm,colback=cellbackground, colframe=cellborder]
\prompt{In}{incolor}{69}{\boxspacing}
\begin{Verbatim}[commandchars=\\\{\}]
\PY{c+c1}{\PYZsh{} The matrix mat is stored in memory like this}
\PY{n}{mat}\PY{o}{.}\PY{n}{reshape}\PY{p}{(}\PY{o}{\PYZhy{}}\PY{l+m+mi}{1}\PY{p}{,} \PY{n}{order}\PY{o}{=}\PY{l+s+s1}{\PYZsq{}}\PY{l+s+s1}{C}\PY{l+s+s1}{\PYZsq{}}\PY{p}{)}
\end{Verbatim}
\end{tcolorbox}

            \begin{tcolorbox}[breakable, size=fbox, boxrule=.5pt, pad at break*=1mm, opacityfill=0]
\prompt{Out}{outcolor}{69}{\boxspacing}
\begin{Verbatim}[commandchars=\\\{\}]
array([0, 1, 2, 3, 4, 5])
\end{Verbatim}
\end{tcolorbox}
        
    \begin{tcolorbox}[breakable, size=fbox, boxrule=1pt, pad at break*=1mm,colback=cellbackground, colframe=cellborder]
\prompt{In}{incolor}{70}{\boxspacing}
\begin{Verbatim}[commandchars=\\\{\}]
\PY{c+c1}{\PYZsh{} ... and NOT like this}
\PY{n}{mat}\PY{o}{.}\PY{n}{reshape}\PY{p}{(}\PY{o}{\PYZhy{}}\PY{l+m+mi}{1}\PY{p}{,} \PY{n}{order}\PY{o}{=}\PY{l+s+s1}{\PYZsq{}}\PY{l+s+s1}{F}\PY{l+s+s1}{\PYZsq{}}\PY{p}{)}      \PY{c+c1}{\PYZsh{} use order=\PYZsq{}F\PYZsq{} to convert to column\PYZhy{}major storage order}
\end{Verbatim}
\end{tcolorbox}

            \begin{tcolorbox}[breakable, size=fbox, boxrule=.5pt, pad at break*=1mm, opacityfill=0]
\prompt{Out}{outcolor}{70}{\boxspacing}
\begin{Verbatim}[commandchars=\\\{\}]
array([0, 3, 1, 4, 2, 5])
\end{Verbatim}
\end{tcolorbox}
        
    While this is not particularly important initially, as an advanced user
you should remember that you never want to perform on non-contiguous
blocks of memory. This can have devastating effects on performance!

    \begin{tcolorbox}[breakable, size=fbox, boxrule=1pt, pad at break*=1mm,colback=cellbackground, colframe=cellborder]
\prompt{In}{incolor}{71}{\boxspacing}
\begin{Verbatim}[commandchars=\\\{\}]
\PY{c+c1}{\PYZsh{} Avoid operations on non\PYZhy{}contiguous array sections such as}
\PY{n}{mat}\PY{p}{[}\PY{p}{:}\PY{p}{,} \PY{l+m+mi}{1}\PY{p}{]}

\PY{c+c1}{\PYZsh{} Contiguous array sections are fine}
\PY{n}{mat}\PY{p}{[}\PY{l+m+mi}{1}\PY{p}{]}
\end{Verbatim}
\end{tcolorbox}

            \begin{tcolorbox}[breakable, size=fbox, boxrule=.5pt, pad at break*=1mm, opacityfill=0]
\prompt{Out}{outcolor}{71}{\boxspacing}
\begin{Verbatim}[commandchars=\\\{\}]
array([3, 4, 5])
\end{Verbatim}
\end{tcolorbox}
        
    \begin{center}\rule{0.5\linewidth}{0.5pt}\end{center}

\hypertarget{exercises}{%
\section{Exercises}\label{exercises}}

TBA

    \hypertarget{solutions}{%
\section{Solutions}\label{solutions}}

TBA


    % Add a bibliography block to the postdoc
    
    
    
\end{document}
